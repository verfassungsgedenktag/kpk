\subsection{Произведение рядов, отсортированное произведение}

\begin{ndefinition}
	Возьмём биекцию \(\gamma \colon \mathbb{N} \to \mathbb{N} \times \mathbb{N}\), которая переводит \(x\) в \((\varphi(x), \psi(x))\). \textit{Произведением} рядов \(\sum a_k\), \(\sum b_k\) называется ряд вида \(\sum_{k=1}^{+\infty} a_{\varphi(k)} b_{\psi(k)}\).
\end{ndefinition}

\begin{ndefinition}
	Пусть \(\sum a_k \cdot \sum b_k = \sum c_k\), где \(c_k = a_0 b_k + a_1 b_{k-1} + \ldots + a_k b_0\). Тогда ряд \(\sum c_k\) называется \textit{отсортированным произведением}.
\end{ndefinition}

\subsection{Перестановка ряда}

\begin{definition}
	Рассмотрим биекцию \(\omega \colon \mathbb{N} \to \mathbb{N}\). Ряд \(\sum b_k\) называется \textit{перестановкой} ряда \(\sum a_k\), если для всех \(k\) \(b_k = a_{\omega(k)}\). 
\end{definition}

\subsection{Бесконечное произведение}

\begin{definition}
	Выражение \(\prod\limits_{k=1}^{+\infty} a_k\) называется \textit{бесконечным произведением}. Обозначим \(P_n = \prod\limits_{k=1}^{n} a_k\). Если существует конечный предел \(\lim\limits_{n \to \infty} P_n\), то говорят, что произведение \textit{сходится}; если \(\lim\limits_{n \to \infty} P_n = 0\), говорят, что оно расходится к нулю. Во всех остальных случаях (предел равен бесконечности или не существует) говорят, что произведение расходится.
\end{definition}

\begin{remark} \hypertarget{besk}{}
	Пусть для всех \(k\) \(a_k \neq 0\). Обозначим \(\Pi_n = \prod\limits_{k=n}^{+\infty} a_k\) и отметим некоторые свойства бесконечного произведения:
	\begin{enumerate}
		\item \(\prod\limits_{k=1}^{+\infty} a_k = \prod\limits_{k=1}^{n-1} a_k \cdot \Pi_n\), и сходимость левой части равносильна сходимости правой для всех \(n\),
		\item \(\prod a_k\) сходится \(\Rightarrow\) \(\Pi_n \xrightarrow[n \to \infty]{} 1\),
		\item \(\prod a_k\) сходится \(\Rightarrow\) \(a_k \to 1\),
		\item Сходимость \(\prod a_k\) возможна только если, НСНМ, \(a_k > 0\),
		\item Пусть \(a_k > 0\). Тогда \(\prod a_k\) сходится \(\Leftrightarrow\) ряд \(\sum \ln a_k\) сходится. Причём если \(\sum \ln a_k = S\), то \(\prod a_k = e^S\).
	\end{enumerate}
\end{remark}
\begin{proof}
	Докажем, но не очень подробно:
	\begin{enumerate}
		\item Просто перейдём к пределу при \(n \to \infty\). Если предел слева существует, то и справа тоже, и наоборот,
		\item Следует из того, что \(\Pi_n = \frac{\prod a_k}{P_{n-1}} \xrightarrow[n \to \infty]{} 1\),
		\item Так как \(a_k = \frac{P_k}{P_{k-1}} \xrightarrow[k \to \infty]{} 1\),
		\item По теореме о стабилизации знака,
		\item Очевидно из того, что 
		\begin{gather*}
			P_n = a_1 \cdot a_2 \cdot \ldots \cdot a_n \xrightarrow[n \to \infty]{} P, \\
			\ln a_1 + \ln a_2 + \ldots + \ln a_n \xrightarrow[n \to \infty]{} \ln P.
		\end{gather*}
	\end{enumerate}
\end{proof}

\subsection{Скалярное произведение, евклидова норма и метрика в \(\mathbb{R}^m\)}

\begin{definition}
	Напомним, что \(\mathbb{R}^m\) --- линейное пространтсво вида \linebreak \(\{(x_1, \ldots, x_m) \mid x_i \in \mathbb{R}\}\). Также напомним некоторые понятия:
	\begin{description}
		\item[Скалярным произведением] \(\langle x, y \rangle\) векторов \(x, y \in \mathbb{R}^m\) называется выражение вида \( \sum\limits_{i=1}^n x_i y_i\).
		\item[Евклидовой нормой] \(||x||\) вектора \(x \in \mathbb{R}^m\) называется выражение вида \linebreak \(\sqrt{\langle x, x \rangle} = \sqrt{x_1^2 + \ldots + x_m^2}\).
		\item[Метрикой] \(\rho\) в \(\mathbb{R}^m\) будем называть отображение вида \(\rho(x, y) = || x - y||\). 
	\end{description}
\end{definition}

\begin{remark}[принцип экономии палочек]
	Далее норму вектора \(x\) будем обозначать как  \(|x|\), а не \(||x||\).
\end{remark}

\subsection{Окрестность точки в \(\mathbb{R}^m\), открытое множество}

\begin{definition}
	Введём следующие понятия:
	\begin{description}
		\item[Шаром] \(B_r (a)\) с центром в точке \(a\) и радиусом \(r\) называется множество вида \(\{x \mid \rho(x, a) < r\}\).
		\item[Эпсилон-окрестностью] \(U(a)\) точки \(a\)называется шар \(B_\varepsilon (a)\).
		\item[Открытым множеством] назовём множество, каждая точка которого является \textit{внутренней}, то есть входит в него с некоторой окрестностью.
	\end{description}
\end{definition}

\subsection{\itshape Сходимость последовательности в \(\mathbb{R}^m\), покоординатная сходимость}

\begin{definition}
	Последовательность  \((x^{(n)})\) (нижний индекс зарезервируем для номера координаты) \textit{сходится} к \(a\), если \[
		\forall \varepsilon > 0 \quad \exists N \quad \forall n > N \quad \rho(x^{(n)}, a) < \varepsilon.
	\]
	
	Это, очевидно, равносильно \textit{покоординатной сходимости \(x^{(n)}\) к \(a\)}, то есть тому, что для любого \(k \in 1 : m \enskip x_k^{(n)} \to a_k\).
\end{definition}

\subsection{\itshape Предельная точка, замкнутое множество, замыкание}

\begin{definition}
	Точка  \(a\) называется \textit{предельной точкой множества} \(F\), если \[
		\forall \varepsilon > 0 \quad \dot B_\varepsilon (a) \cap F \neq \varnothing.
	\]
	
	Множество называется \textit{закрытым}, если оно содержит все свои предельные точки.
	
	\textit{Замыканием} множества \(G\) называется минимальное по включению закрытое множество, содержащее \(G\).
\end{definition}

\subsection{Координатная функция}

\begin{definition}
	Рассмотрим функцию \(f \colon \mathbb{R}^m \to \mathbb{R}^m\) следующего вида: \(f(x) = (f_1 (x), \ldots, f_m (x))\). Функции \(f_1 (x), \ldots, f_m (x)\) называются \textit{координатными}.
\end{definition}

\begin{remark}
	\textit{Сходимость} и \textit{покоординатная сходимость} функции \(f\) определяются привычным образом.
\end{remark}

\subsection{Двойной предел, повторный предел}

\begin{definition}
	Пусть \(D_1, D_2 \subset \mathbb{R}\), \(a_1\) --- предельная точка \(D_1\), \(a_2\) --- предельная точка \(D_2\). Обозначим \(D \supset (D_1 \setminus \{a_1\}) \times (D_2 \setminus \{a_2\})\).
	
	Теперь рассмотрим функцию \(f \colon D \to \mathbb{R}\). Тогда:
	\begin{enumerate}
		\item Если \[
			\forall x_1 \in D_1 \setminus \{a_1\} \quad \exists \varphi(x_1) := \lim_{x_2 \to a_2} f (x_1, x_2) \in \mathbb{R},
		\]
		то \(\lim\limits_{x_1 \to a_1} \varphi(x_1)\) называется \textit{повторным пределом}.
		\item Аналогично если \[
		\forall x_2 \in D_2 \setminus \{a_2\} \quad \exists \psi(x_2) := \lim_{x_1 \to a_1} f (x_1, x_2) \in \mathbb{R},
		\]
		то \(\lim\limits_{x_2 \to a_2} \psi(x_2)\) также называется \textit{повторным пределом}.
		\item Если \[
			\forall U(L) \quad \exists V_1 (a_1), V_2 (a_2) \quad \forall \parbox[t]{3cm}{\(x_1 \in \dot V_1 (a_1) \cap D_1\), \\ \(x_2 \in \dot V_2 (a_2) \cap D_2\)} \quad f(x_1, x_2) \in U(L),
		\]
		то \(L\) называется \textit{двойным пределом}.
	\end{enumerate}
\end{definition}

\subsection{Предел по направлению, предел вдоль пути}

\begin{ndefinition}
	Рассмотрим функцию \(f \colon \mathbb{R}^2 \to \mathbb{R}\) и \(v = (v_1, v_2) \in \mathbb{R}^2\). Предел \(\lim\limits_{t \to +0} f(a_1 + t v_1, a_2 + t v_2) = \lim\limits_{x \to a} f(x) |_L\), где \(L\) --- луч с началом в точке \(a\) и вектором \(v\), называется \textit{пределом \(f\) по направлению}.
\end{ndefinition}

\begin{ndefinition}
	Рассмотрим функцию \(f \colon \mathbb{R}^2 \to \mathbb{R}\) и гладкий путь \(\gamma\), причём для любого \(t\) \(\gamma(t) \neq 0\). Предел \(\lim\limits_{x \to a} f(x) |_{C_\gamma}\), где \(C_\gamma\) --- носитель пути \(\gamma\), называется \textit{пределом \(f\) вдоль пути}.
\end{ndefinition}
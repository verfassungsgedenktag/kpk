\subsection{Неравенство Йенсена для интегралов}

\begin{theorem} \hypertarget{Йенсен-интегралы}{}
	Пусть функция \(f\) выпукла на \(\langle A, B \rangle\). Введём непрерывные функции \(x \colon [a, b] \to \langle A, B \rangle\), \(\alpha \colon [a, b] \to [0, +\infty)\), причём \(\int_a^b \alpha(t) \, dt = 1\). Тогда \[
		f \left(\int_a^b \alpha(t) x(t) \, dt \right) \leqslant \int_a^b \alpha(t) f(x(t)) \, dt.
	\]
\end{theorem}
\begin{proof}
	Для начала проверим, что \(x^* = \int_a^b \alpha(t) x(t) \, dt\) вообще лежит на \(\langle A, B \rangle\). Обозначим \(m = \inf\limits_{[a, b]} x(t), \ M = \sup\limits_{[a, b]} x(t)\). Тогда \[
		m = m \int_a^b \alpha(t) \, dt \leqslant x^* \leqslant M \int_a^b \alpha(t) \, dt = M.
	\]
	Отдельно оговорим, что случаи \(x^* = M = B\) или \(x^* = m = A\) не реализуются на промежутке вида  \(\langle A, B)\) или \((A, B \rangle\) соответственно, так как их реализация предполагает, что функция \(x(t)\) равна \(B\) или \(A\) соответственно, а значит, они включены в область определения.
	
	Теперь, собственно, доказательство. В точке \(x^*\) проведём опорную прямую \(y = p \xi + q\) к графику функции \(f\). Тогда \[
		f(x^*) = px^* + \beta = p \int_a^b \alpha(t) x(t) \, dt + \int_a^b \beta \cdot \alpha(t) = \int_a^b \alpha(t) \cdot (p x(t) + \beta) \, dt.
	\]
	Так как \((px(t) + \beta)\) --- это точка на опорной прямой с абсциссой \(x_k\) в силу того, что \(f\) выпукла, \(px(t) + \beta \leqslant f(x(t))\). Получаем \[
		f(x^*) = \int_a^b \alpha(t) \cdot (p x(t) + \beta) \, dt \leqslant \int_a^b \alpha(t) f(x(t)) \, dt.
	\]
\end{proof}

\subsection{Неравенство Коши (для сумм и интегралов)}

\begin{ntheorem}
	Для любых \(a_1, \ldots, a_n > 0\) выполняется \[
		\sum_{k=1}^n a_k \geqslant \sqrt[n]{a_1 \cdot \ldots \cdot a_n}.
	\]
\end{ntheorem}
\begin{proof}
	Запишем \hyperlink{Йенсен-суммы}{\bfseries неравенство Йенсена} для выпуклой вверх (знак неравенства изменится на противоположный) функции \(\ln x\) и \(\alpha_k = \frac{1}{n}\) для всех \(k\): \[
		\ln \left(\frac{1}{n} \sum_{k=1}^n a_k \right) \geqslant \frac{1}{n} \sum_{k=1}^n \ln a_k = \frac{1}{n} \, \ln (a_1 \cdot \ldots \cdot a_n) = \ln \sqrt[n]{a_1 \cdot \ldots \cdot a_n}.
	\]
	Возьмём экспоненту от левой и правой частей и получим требуемый результат.
\end{proof}

\begin{ntheorem}
	Пусть функция \(f\) непрерывна на \([a, b]\). Пусть также \(f > 0\). Тогда \[
		\frac{1}{b - a} \int_a^b f \geqslant \exp \left(\frac{1}{b - a} \int_a^b \ln f \right).
	\]
\end{ntheorem}
\begin{proof}
	Запишем \hyperlink{Йенсен-интегралы}{\bfseries интегральное неравенство Йенсена} (знак неравенства изменится на противоположный) для выпуклой вверх функции \(\ln x\) и \(\alpha(t) \equiv \dfrac{1}{b - a}\): \[
		\ln \left(\frac{1}{b - a} \int_a^b f(t) \, dt \right) \geqslant \frac{1}{b - a} \int_a^b \ln f(t) \, dt.
	\]
	Возьмём экспоненту от левой и правой частей и получим требуемый результат. 
\end{proof}

МБ ДОДЕЛАТЬ ЧЕРЕЗ ПРЕДЕЛЬНЫЙ ПЕРЕХОД

\subsection{Неравенство Гёльдера для сумм}

\begin{theorem} \hypertarget{Гёльдер-суммы}{}
	Возьмём \(p > 1\) и подберём число \(q\) такое, что \(\dfrac{1}{p} + \dfrac{1}{q} = 1\), то есть \(q = \dfrac{p}{p - 1}\). Тогля для любых \(a_1, \ldots, a_n > 0\) и \(b_1, \ldots, b_n > 0\) выполняется \[
		\sum_{k=1}^n a_k b_k \leqslant \left(\sum_{k=1}^{n} a_k^p \right)^{1/p} \left(\sum_{k=1}^{n} b_k^q \right)^{1/q}\footnote{При \(p = q = 2\) это просто неравенство Коши -- Буняковского.}.
	\]
\end{theorem}
\begin{proof}
	Запишем \hyperlink{Йенсен-суммы}{\bfseries неравенство Йенсена} для функции \(x^p\) (выпуклой при \(p > 1\)):
	\begin{equation} \label{ЙенГёль}
		\left(\sum_{k=1}^n \alpha_k x_k \right)^{p} \leqslant \sum_{k=1}^n \alpha_k x_k^{p}.
	\end{equation}
	В качестве \(\alpha_k\) возьмём выражение \(\dfrac{b_i^q}{\sum b_i^q}\), а в качестве \(x_k\) --- выражение \(a_k b_k^{-\frac{1}{p - 1}} \sum b_i^q\). Тогда
	\begin{gather*}
		\alpha_k x_k = \frac{b_k^q}{\sum b_i^q} \cdot a_k b_k^{-\frac{1}{p - 1}} \textstyle \sum b_i^q = a_k b^{q - \frac{1}{p - 1}} = a_k b_k, \\
		\alpha_k x_k^p = \frac{b_k^q}{\sum b_i^q} \cdot a_k^p b_k^{-\frac{p}{p - 1}} \left(\sum b_i^q \right)^p = a_k^p \left(\sum b_i^q \right)^{p-1}.
	\end{gather*}
	То есть неравенство~\eqref{ЙенГёль} теперь имеет вид \[
		\left(\sum_{k=1}^n a_k b_k \right)^{p} \leqslant \left(\sum_{k=1}^n a_k^p \right) \left(\sum_{i=1}^n b_i^q \right)^{p-1}.
	\]
	Теперь возведём обе части в степень \(1/p\) и получимый необходимый результат.
\end{proof}

\begin{remark}
	Равенство достигается в случае \(x_1 = x_2 = \ldots x_n\), то есть в случае, когда выражение \(a_k b_k^{-\frac{1}{p - 1}}\) не зависит от \(i\). Это равносильно тому, что \(a_k^p = b_k^q \cdot C\), где \(C\) --- некая константа, которой и равно выражение. 
\end{remark}

\subsection{\itshape Неравенство Гёльдера для интегралов}

\begin{theorem}
	Пусть функции \(f, g\) непрерывны на  \([a, b]\), причём \(f, g \geqslant 0\). Возьмём \(p > 1\) и подберём число \(q\) такое, что \(\dfrac{1}{p} + \dfrac{1}{q} = 1\), то есть \(q = \dfrac{p}{p - 1}\). Тогда \[
		\int_a^b fg \leqslant \left(\int_{a}^{b} f^p \right)^{1/p} \left(\int_{a}^{b} g^q \right)^{1/q}.
	\]
\end{theorem}
\begin{proof}
	Возьмём дробление \(\{x_k\}_{k=0}^n\) отрезка  \([a, b]\) и оснащение с точками вида \(\xi_k = x_k = a + \frac{b - a}{n} \, k\), тогда \(\Delta_k = x_{k} - x_{k-1} = \frac{b - a}{n}\), то есть мы разбили \([a, b]\) на \(n\) равных отрезков и взяли за \(\xi_k\) их левый конец.
	
	Запишем неравенство для сумм:
	\begin{equation} \label{ГёльГёль}
		\sum_{k=1}^n a_k b_k \leqslant \left(\sum_{k=1}^{n} a_k^p \right)^{1/p} \left(\sum_{k=1}^{n} b_k^q \right)^{1/q}
	\end{equation}
	и обозначим \(a_k = f(x_k) \cdot \Delta_k^{1/p}\), \(b_k = g(x_k) \cdot \Delta_k^{1/q}\). Тогда
	\begin{gather*}
		\sum_{k=1}^n a_k b_k = \sum_{k=1}^n f(x_k) g(x_k) \Delta_k^{1/p + 1/q} = \sum_{k=1}^n f(x_k) g(x_k) \Delta_k \xrightarrow[n \to \infty]{} \int_a^b fg, \\
		\left(\sum_{k=1}^n a_k^p \right)^{1/p} = \left(\sum_{k=1}^n f^p(x_k) \Delta_k \right)^{1/p} \xrightarrow[n \to \infty]{} \left(\int_a^b f^p \right)^{1/p}, \\
		\left(\sum_{k=1}^n b_k^q \right)^{1/q} = \left(\sum_{k=1}^n g^q(x_k) \Delta_k \right)^{1/q} \xrightarrow[n \to \infty]{} \left(\int_a^b g^q\right)^{1/p}.
	\end{gather*}
	И мы уже получили, что требовалось.
\end{proof}

\subsection{Неравенство Минковского}

\begin{ntheorem}
	Отображение \(\displaystyle (x_1, \ldots, x_n) \mapsto \left(\sum_{k=1}^n |x_k|^p \right)^{1/p}\) при \(p \geqslant 1\) является нормой. То есть оно невырожденно, однорородно (это легко проверить), а также для него выполняется неравенство треугольника, то есть \[
		\left(\sum_{k=1}^n |a_k + b_k|^p \right)^{1/p} \leqslant \left(\sum_{k=1}^n |a_k|^p \right)^{1/p} + \left(\sum_{k=1}^n |b_k|^p \right)^{1/p}
	\]
	(что, собственно, и является утверждением теоремы и называется неравенством Минковского).
\end{ntheorem}
\begin{proof}
	Заметим, что при \(p = 1\) утверждение очевидно, так как превращается в самое обычное неравенство треугольника для самой обычной нормы \(|x|\). Рассмотрим \(p > 1\).
	
	Рассмотрим выражение \(\sum |a_k| \cdot |a_k + b_k|^{p-1}\) и запишем для него \hyperlink{Гёльдер-суммы}{неравенство Гёльдера}:
	\begin{multline*}
		\sum_{k=1}^n |a_k| \cdot |a_k + b_k|^{p-1} = \left(\sum_{k=1}^n |a_k|^p \right)^{1/p} \left(\sum_{k=1}^n |a_k + b_k|^{(p - 1) \cdot \frac{p}{p-1}} \right)^{\frac{p-1}{p}} = \\
		= \left(\sum_{k=1}^n |a_k|^p \right)^{1/p} \left(\sum_{k=1}^n |a_k + b_k|^p \right)^{\frac{p-1}{p}}.
	\end{multline*}
	Аналогично
	\begin{multline*}
		\sum_{k=1}^n |b_k| \cdot |a_k + b_k|^{p-1} \leqslant \left(\sum_{k=1}^n |b_k|^p \right)^{1/p} \left(\sum_{k=1}^n |a_k + b_k|^{(p - 1) \cdot \frac{p}{p-1}} \right)^{\frac{p-1}{p}} = \\
		= \left(\sum_{k=1}^n |b_k|^p \right)^{1/p} \left(\sum_{k=1}^n |a_k + b_k|^p \right)^{\frac{p-1}{p}}.
	\end{multline*}
	Сложим получившиеся неравенства:
	\begin{multline*}
		\sum_{k=1}^n (|a_k| + |b_k|) \cdot |a_k + b_k|^{p-1} \leqslant \\
		\leqslant \left(\left(\sum_{k=1}^n |a_k|^p \right)^{1/p} + \left(\sum_{k=1}^n |b_k|^p \right)^{1/p} \right) \left(\sum_{k=1}^n |a_k + b_k|^p \right)^{\frac{p-1}{p}}.
	\end{multline*}
	В силу неравенства треугольника для \(|x|\) справедливо 
	\begin{multline*}
		\sum_{k=1}^n |a_k + b_k|^p \leqslant \sum_{k=1}^n (|a_k| + |b_k|) \cdot |a_k + b_k|^{p-1} \leqslant \\
		\leqslant \left(\left(\sum_{k=1}^n |a_k|^p \right)^{1/p} + \left(\sum_{k=1}^n |b_k|^p \right)^{1/p} \right) \left(\sum_{k=1}^n |a_k + b_k|^p \right)^{\frac{p-1}{p}}.
	\end{multline*}
	И теперь разделим неравенство на \((\sum |a_k + b_k|^p)^{\frac{p-1}{p}}\): \[
		\left(\sum_{k=1}^n |a_k + b_k|^p\right)^{1/p} \leqslant \left(\sum_{k=1}^n |a_k|^p \right)^{1/p} + \left(\sum_{k=1}^n |b_k|^p \right)^{1/p}.
	\]
\end{proof}

\begin{ntheorem}
	Пусть функции \(f, g\) непрерывны на \([a, b]\). Тогда при \(p \geqslant 1\) справедливо \[
		\left(\int_a^b |f + g|^p \right)^{1/p} \leqslant \left(\int_a^b |f|^p \right)^{1/p} + \left(\int_a^b |g|^p \right)^{1/p}.
	\]
\end{ntheorem}
\begin{proof}
	Упражнение...
\end{proof}

МБ ДОДЕЛАТЬ ИНТЕГРАЛЬНУЮ ВЕРСИЮ

\subsection{Теорема об условиях сходимости бесконечного произведения}

ДОДЕЛАТЬ

\subsection{Лемма об оценке приближения экспоненты ее замечательным пределом}

ДОДЕЛАТЬ

\subsection{Формула Эйлера для \(\Gamma\)-функции}

ДОДЕЛАТЬ

\subsection{Формула Вейерштрасса для \(\Gamma\)-функции}

ДОДЕЛАТЬ

\subsection{Вычисление произведений с рациональными сомножителями}

ДОДЕЛАТЬ

\subsection{Длина гладкого пути}

\begin{definition}
	Функция \(l\), заданная на множестве всех гладкий путей, называется \textit{длиной гладкого пути}, если она обладает следуюшими свойствами:
	\begin{enumerate}
		\item \(l \geqslant 0\),
		\item \label{way2} \(l\) --- аддитивна, то есть если мы возьмём произвольный путь \(\gamma \colon [a, b] \to \mathbb{R}^m\) и произвольную точку \(c \in (a, b)\) и рассмотрим функции \(\gamma_1\) --- сужение \(\gamma\) на отрезок \([a, c]\), и \(\gamma_2\) --- сужение \(\gamma\) на отрезок \([c, b]\), то \[
			l(\gamma) = l(\gamma_1) + l(\gamma_2).
		\]
		\item \label{way3}Если носитель \(C_{\widetilde{\gamma}}\) пути \(\widetilde{\gamma}\) является образом сжатия носителя \(C_\gamma\) какого-то пути \(\gamma\), то длина 	\(\widetilde{\gamma}\) не больше длины \(\gamma\), то есть если \[
			\exists T \colon C_\gamma \xrightarrow[\text{на}]{} C_{\widetilde{\gamma}}, \ \textit{такая, что} \ \forall x, y \in C_\gamma \quad \rho(x, y) \geqslant \rho(T(x), T(y)),
		\]
		то \(l(\widetilde{\gamma}) \leqslant l(\gamma)\).
		\item Нормировка: если \(\gamma \colon [0, 1] \to \mathbb{R}^m\) --- линейный путь из \(A\) в \(B\), то есть \(\gamma(t) = (1 - t)A + tB\), то \(l(\gamma) = \rho(A, B)\).
	\end{enumerate}
\end{definition}

\begin{remark}
	Отметим некоторые свойства длины:
	\begin{enumerate}
		\item Из аксиомы~\ref{way3} следует, что длина дуги больше длины хорды,
		\item При растяжении длина растёт,
		\item При движении длина пути не меняется.
	\end{enumerate}
\end{remark}

\subsection{Вариация функции на промежутке}

\begin{definition}
	Пусть \(f \colon [a, b] \to \mathbb{R}\). Выберём дробление \(\tau = \{t_i\}_{i = 0}^n\) отрезка \([a, b]\). \textit{Вариацией}  функции \(f\) на отрезке \([a, b]\) называется величина \[
		\bigvee_a^b f = \sup_\tau \sum_{i = 0}^n |f(t_{i + 1}) - f(t_i)|.
	\]
\end{definition}

\begin{remark} \hypertarget{ogrvar}{}
	Если \(\bigvee\limits_a^b f < +\infty\), то \(f\) называется функцией \textit{ограниченной вариации}.
\end{remark}

\subsection{Эпсилон-сеть, сверхограниченное множество}

\begin{ndefinition}
	Множество \(E \subset X\)  \textit{\(\varepsilon\)-сетью} для \(D\), если \[
		\forall x \in D \quad \exists y \in E \quad \rho(x, y) < \varepsilon.
	\]
\end{ndefinition}

\begin{ndefinition}
	Множество \(D\) называется \textit{сверхограниченным} в \(X\), если для любого положительного \(\varepsilon\) существует конечная \(\varepsilon\)-сеть.
\end{ndefinition}

\subsection{\itshape Несобственный интеграл, сходимость, расходимость}

\begin{definition}
	Рассмотрим функцию \(f \colon [a, b) \to \mathbb{R}\), которая является кусочно-непрерывной на отрезке \([a, A]\) для любого \(A \in (a, b)\) (назовём такую функцию \textit{допустимой}). Символ \(\int_a^{\to b} f\) называют \textit{несобственным интегралом}. По определению \[
		\int_a^{\to b} f = \lim_{A \to b - 0} \int_a^A f,
	\]
	если предел существует в \(\overline{\mathbb{R}}\). Если предел принадлежит \(\mathbb{R}\), говорят, что несобственный интеграл \textit{сходится}; в противном случае говорят, что он \textit{расходится}.
\end{definition}

\subsection{Критерий Больцано -- Коши сходимости несобственного интеграла}

\begin{theorem} \hypertarget{Критерий Больцано -- Коши сходимости несобственного интеграла}{}
	Пусть функция \(f\) допустима. Интеграл \(\int_a^b f\) сходится тогда и только тогда, когда \[
		\forall \varepsilon > 0 \quad \exists \delta \in (a, b) \quad \forall A, B \in (\delta, b) \quad \left|\int_A^B f \right| < \varepsilon.
	\]
\end{theorem}

\subsection{\itshape Гамма-функция Эйлера}

\begin{definition}
	Функция \(\displaystyle \Gamma(t) = \int_{0}^{+\infty} x^{t - 1} e^{-x} \, dx\) называется \textit{гамма-функцией Эйлера} (причём интеграл сходится при \(t > 0\)).
\end{definition}

\subsection{\itshape Абсолютно сходящиеся интеграл, ряд}

\begin{ndefinition}
	Пусть функция \(f\) допустима на \([a, b)\). Интеграл \(\int_{a}^{\to b} f\) называется \textit{абсолютно сходящимся}, если выполяются следующие условия:
	\begin{enumerate}
		\item \(\int_{a}^{\to b} f\) сходится,
		\item \(\int_{a}^{\to b} |f|\) тоже сходится.
	\end{enumerate}
\end{ndefinition}

\begin{ndefinition}
	Аналогично ряд \(\sum a_k\) называется \textit{абсолютно сходящимся}, если:
	\begin{enumerate}
		\item Ряд \(\sum a_k\) сходится,
		\item Ряд \(\sum |a_k|\) тоже сходится.
	\end{enumerate}
\end{ndefinition}

\subsection{\itshape Числовой ряд, сумма ряда, сходимость, расходимость}

\begin{definition}
	Введём следующие понятия:
	\begin{description}
		\item[Числовым рядом] называется выражение вида \(\sum\limits_{k=1}^{+\infty} a_k\) (иногда для краткости будем писать \(\sum a_k\) или даже \((A)\)). Такой объект также называют \textit{формальным рядом}.
		\item[Частичной суммой ряда] называется выражение вида \(S_N = \sum\limits_{k=1}^{N} a_k\) (иногда, если хотим отметить, каким символом обозначаем слагаемое ряда, будем писать \(S_N^a\)).
		\item[Сходящимся] называют ряд, для которого существует конечный предел частичных сумм \(\lim\limits_{N \to \infty} S_N = S\) (тогда \(S\) называют \textit{суммой ряда}).
		\item[Расходящимся] называют ряд, для которого справедливо обратное, то есть предела частичных сумм либо не существует, либо он равен бесконечности.
	\end{description}
\end{definition}

\begin{nremark}
	Начинать нумерацию можно не с единицы, а с любого целого числа \(A\). Тогда переопределим \textit{частичную сумму} как \(S_N = \sum\limits_{k=A}^{N} a_k\).
\end{nremark}

\begin{nremark}
	Очевидно, что \(S_N - S_{N-1} = a_N\). Это значит, что подбирать слагаемые, чтобы получить ряд с нужной частичной суммой, очень просто!
\end{nremark}

\subsection{\(N\)-й остаток ряда}

\begin{definition}
	Выражение вида \(R_N = \sum\limits_{k=N}^{+\infty} a_k\) называется \textit{\(N\)-м остатком ряда} \(\sum a_k\).
\end{definition}

\subsection{Критерий Больцано -- Коши сходимости числового ряда}

\begin{theorem}
	Ряд \(\sum a_k\) сходится тогда и только тогда, когда \[
	\forall \varepsilon > 0 \quad \exists N \quad \forall n > N \quad \forall p \in \mathbb{N} \quad |a_{n+1} + a_{n+2} + \ldots + a_{n+p}| < \varepsilon.
	\]
\end{theorem}
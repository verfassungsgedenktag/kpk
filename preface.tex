\anonsection{Предисловие}

Этот конспект был создан при подготовке к допсе по матанализу за второй семестр. Теоремы, определения и формулировки упорядочены в соответствии со списком вопросов от Кохася. При написании использовались записи лекциий КПК в ИТМО (\href{https://youtube.com/playlist?list=PLd7QXkfmSY7avIqoacyLFhwAAfO-JQX7w&si=zLviQ8XKvk3Mrje3}{вот} и \href{https://youtube.com/playlist?list=PLd7QXkfmSY7YppM1nXNRKzXeQN0TYfFBQ&si=TGkLA1tfMZIdTC02}{вот}), ХолиКПК за первый и второй семестры (\href{https://github.com/snitron/ct-itmo}{репозиторий}), конспект А. Флоринского за третий семестр (\href{https://github.com/Alexandr5476/Matanaliz}{репозиторий}), учебник Виноградова и рукописные конспекты (собственные и несобственные).

\paragraph{Условные обозначения}

\textcolor{blue}{Синим} цветом выделены ссылки на что-то внутри документа. \textcolor{magenta}{Розовым} --- ссылки на что-то вне документа. \textit{Курсивом} отмечены темы, которые необходимо (но не достаточно) знать для получения положительной оценки (а иногда курсив используется просто для выделения в тексте ключевых понятий). \textcolor{cyan}{Голубой} тоже иногда используется, чтобы выделить что-то в тексте. \textcolor{red}{Красным} помечены темы, частично или полностью отсутствующие в конспекте (по некоторым из них есть мелкие закомментированные наработки) --- будет круто, если вы за\TeX аете и пришлёте их!

\paragraph{Обратная связь} Просьба все ошибки, опечатки, претензии и предложения присылать в тг (@verfassungsgedenktag) или создавать pull request (\href{https://github.com/verfassungsgedenktag/kpk}{репозиторий}).

\subsection{\itshape Гамма-функция Эйлера, простейшие свойства}

\begin{theorem}
	Изучим \(\Gamma(t)\) на предмет наличия всяких замечательных \linebreak свойств:
	\begin{enumerate}
		\item \(\Gamma(t)\) сходится при \(t > 0\) и расходится в противном случае,
		\item \(\Gamma(t)\) выпукла,
		\item Для любых \(n \in \mathbb{Z}_+\) \(\Gamma(n + 1) = n!\),
		\item График,
		\item \(\Gamma \left(\dfrac{1}{2} \right) = \sqrt{\pi}\).
	\end{enumerate}
\end{theorem}

\begin{proof}
	Докажем:
	\begin{enumerate}
		\item Заметим, что при \(x \to 0\) подынтегральное выражение \(x^{t - 1} e^{-x}\) эквивалентно \(x^{t - 1} = \dfrac{1}{x^{1 - t}}\). Соответственно, при \(1 - t \geqslant 1\), то есть при \(t \leqslant 0\) интеграл расходится.
		
		Проверим, при всех ли других значениях \(t\) он сходится. Запишем подынтегральное выражение \(x^{t - 1} e^{-x}\) как \(x^{t - 1} e^{-\frac{x}{2}} e^{-\frac{x}{2}}\). Так как показательная функция \(e^{-\frac{x}{2}}\) при росте \(x\) убывает быстрее, чем растёт степенная \(x^{t - 1}\), выражение \(x^{t - 1} e^{-\frac{x}{2}}\) стремится к нулю, а значит \[
			x^{t - 1} e^{-\frac{x}{2}} e^{-\frac{x}{2}} \leqslant  e^{-\frac{x}{2}}.
		\]
		Интеграл от \(e^{-\frac{x}{2}}\) сходится как эталонный, а значит по \hyperlink{priz}{признаку сравнения} \(\Gamma(t)\) тоже сходится.
	\end{enumerate}
\end{proof}

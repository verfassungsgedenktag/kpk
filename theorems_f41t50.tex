\subsection{\itshape Гамма-функция Эйлера, простейшие свойства}

\begin{theorem} \hypertarget{t41}{}
	Изучим \(\Gamma(t)\) на предмет наличия всяких замечательных \linebreak свойств:
	\begin{enumerate}
		\item \(\Gamma(t)\) сходится при \(t > 0\) и расходится в противном случае,
		\item \(\Gamma(t)\) выпукла,
		\item \(\Gamma(t + 1) = t \cdot \Gamma(t)\),% Для любых \(n \in \mathbb{Z}_+\) \(\Gamma(n + 1) = n!\),
		\item График,
		\item \(\Gamma \left(\frac{1}{2} \right) = \sqrt{\pi}\).
	\end{enumerate}
\end{theorem}
\begin{proof}
	Докажем:
	\begin{enumerate}
		\item Заметим, что при \(x \to 0\) подынтегральное выражение \(x^{t - 1} e^{-x}\) эквивалентно \(x^{t - 1} = \dfrac{1}{x^{1 - t}}\). Соответственно, при \(1 - t \geqslant 1\), то есть при \(t \leqslant 0\) интеграл расходится.
		
		Проверим, при всех ли других значениях \(t\) он сходится. Запишем подынтегральное выражение \(x^{t - 1} e^{-x}\) как \(x^{t - 1} e^{-\frac{x}{2}} e^{-\frac{x}{2}}\). Так как показательная функция \(e^{-\frac{x}{2}}\) при росте \(x\) убывает быстрее, чем растёт степенная \(x^{t - 1}\), выражение \(x^{t - 1} e^{-\frac{x}{2}}\) стремится к нулю, а значит \[
			x^{t - 1} e^{-\frac{x}{2}} e^{-\frac{x}{2}} \leqslant  e^{-\frac{x}{2}}.
		\]
		Интеграл от \(e^{-\frac{x}{2}}\) сходится как эталонный, а значит по \hyperlink{priz}{признаку сравнения} \(\Gamma(t)\) тоже сходится.
		\item Рассмотрим подыинтегральное выражение \(x^{t - 1} e^{-x}\) относительно \(t\). При любом \(x\) оно представляет собой показательную функцию \(x^{t - 1}\) с каким-то коэффициентом \(e^{-x}\).
		Показательная функция выпуклая, то есть по определению для любых \(t_1, t_2\) и \(\alpha \in (0, 1)\) \[
			x^{\alpha t_1 + (1 - \alpha) t_2 - 1} e^{-x} \leqslant \alpha x^{t_1 - 1} e^{-x} + (1 - \alpha) x^{t_2 - 1} e^{-x}.
		\]
		Проинтегрируем неравенство по \(x\) на промежутке \((0, +\infty)\): 
		\begin{multline*}
			\int_0^{+\infty} x^{\alpha t_1 + (1 - \alpha) t_2 - 1} e^{-x} \, dx \leqslant \\
			\leqslant \alpha \int_0^{+\infty} x^{t_1 - 1} e^{-x} \, dx + (1 - \alpha) \int_0^{+\infty} x^{t_2 - 1} e^{-x} \, dx,
		\end{multline*}
		что является уже определением выпуклости, записанным для \(\Gamma(t)\).
		\begin{remark}
			Из выпуклости функции \(\Gamma\) следует её непрерывность и даже почти дифференцируемость.
		\end{remark}
		\item Проинтегрируем \(\Gamma(t + 1)\) по частям: \[
			\int_{0}^{+\infty} x^t e^{-x} \, dx = \int_{0}^{+\infty} x^t \, d(-e^{-x}) = -x^t e^{-x} \bigg|_0^{+\infty} + t \cdot \int_{0}^{+\infty} x^{t - 1} e^{-x} \, dx.
		\]
		Двойная подстановка зануляется, так как выражение \(-A^t e^{-A}\) стремится к нулю при \(A \to \infty\), а значит, получаем \[
			\Gamma(t + 1) = t \cdot \int_{0}^{+\infty} x^{t - 1} e^{-x} \, dx = t \cdot \Gamma(t).
		\]
		\begin{remark}
			Легко проверить, что \(\Gamma(1) = 1\). Отсюда получаем, что \(\Gamma(n + 1) = n \cdot \Gamma(n) = n \cdot (n - 1) \cdot \Gamma(n - 1) = \ldots = n!\).
		\end{remark}
		\item Пользуясь соображениями из предыдущего замечания, а также тем фактом, что \(\Gamma(t) = \dfrac{\Gamma(t + 1)}{t}\) ведёт себя как \(\dfrac{1}{t}\) при \(t \to 0\), построим график.
		
		ДОДЕЛАТЬ ГРАФИК
		\item Запишем \[
			\Gamma \left(\frac{1}{2} \right) = \int_{0}^{+\infty} x^{-\frac{1}{2}} e^{-x} \, dx
		\]
		и сделаем замену \(x = y^2\): \[
			\int_{0}^{+\infty} x^{-\frac{1}{2}} e^{-x} \, dx = \int_{0}^{+\infty} \frac{2y}{y} \, e^{-y^2} \, dy = 2 \int_{0}^{+\infty} e^{-y^2} \, dy.
		\]
		Получили \hyperlink{puas}{интеграл Эйлера -- Пуассона}, то есть \(\Gamma \left(\frac{1}{2} \right) = 2 \, \dfrac{\sqrt{\pi}}{2} = \sqrt{\pi}\).
	\end{enumerate}
\end{proof}

\subsection{Теорема об абсолютно сходящихся рядах и интегралах}

\begin{ntheorem} \hypertarget{t42}{}
	Пусть функция \(f\) допустима на \([a, b)\). Тогда следующие утверждения эквивалентны:
	\begin{enumerate}
		\item \(\int_{a}^{b} f\) абсолютно сходится,
		\item \(\int_{a}^{b} |f|\) сходится,
		\item \(\int_{a}^{b} f^+\) и \(\int_{a}^{b} f^-\) сходятся.
	\end{enumerate} 
\end{ntheorem}
\begin{proof}
	Докажем переходы:
	\begin{description}
		\item[\(1 \Rightarrow 2\):] По определению абсолютно сходящегося интеграла.
		\item[\(2 \Rightarrow 3\):] Так как по определению \(f^+ = \max (f, 0)\), \(f^- = \max (-f, 0)\), то \(f^+ \leqslant |f|\), \(f^- \leqslant |f|\), по \hyperlink{priz}{\bfseries признаку сравнения} из сходимости \(\int_{a}^{b} |f|\) следует сходимость интегралов от \(f^+, f^-\).
		\item[\(3 \Rightarrow 1\):] Так как \(f = f^+ - f^-, |f| = f^+ + f^-\), то по \hyperlink{svva}{\bfseries простейшим свойствам несобственного интеграла} из сходимости \(\int_{a}^{b} f^+, \int_{a}^{b} f^-\) следует сходимость интегралов \(\int_{a}^{b} f, \int_{a}^{b} |f|\).
	\end{description}
\end{proof}

\begin{ntheorem}[из будущего]
	Следующие утверждения эквивалентны:
	\begin{enumerate}
		\item Ряд \(\sum a_k\) абсолютно сходится,
		\item Ряд \(\sum |a_k|\) сходится,
		\item Ряды \(\sum a_k^+\) и \(\sum a_k^-\) сходятся
	\end{enumerate}
	\big(\(\sum a_k^+\) и \(\sum a_k^-\) определены как в доказательстве \hyperlink{теорема о перестановке слагаемых}{\bfseries теоремы о перестановке слагаемых}\big).
\end{ntheorem}
\begin{proof}
	Докажем переходы:
	\begin{description}
		\item[\(1 \Rightarrow 2\):] По определению абсолютно сходящегося ряда.
		\item[\(2 \Rightarrow 3\):] Так как по определению \(a_k^+ = \max (a_k, 0)\), \(a_k^- = \max (-a_k, 0)\), то \(a_k^+ \leqslant |f|\), \(a_k^- \leqslant |f|\), по \hyperlink{признак сравнения рядов}{\bfseries признаку сравнения} из сходимости \(\sum |a_k|\) следует сходимость рядов \(\sum a_k^+, \sum a_k^-\).
		\item[\(3 \Rightarrow 1\):] Так как \(\sum a_k = \sum a_k^+ - \sum a_k^-, \sum |a_k| = \sum a_k^+ + \sum a_k^-\), то по \hyperlink{свойства рядов}{\bfseries свойствам рядов} из сходимости \(\sum a_k^+, \sum a_k^-\) следует сходимость рядов \(\sum a_k, \sum |a_k|\).
	\end{description}
\end{proof}

\subsection{Изучение интеграла $\int_1^{\infty} \frac{\sin x}{x^p} \, dx$ на сходимость и абсолютную сходимость}

\begin{example}
	Выясним, при каких \(p \in \mathbb{R}\) интеграл \(\displaystyle \int_1^{\infty} \frac{\sin x}{x^p} \, dx\) сходится и абсолютно сходится.
	
	Сначала о сходимости. Проверим, что при \(p > 0\) интеграл сходится. Проинтегрируем его по частям: \[
		\int_1^{\infty} \frac{\sin x}{x^p} \, dx = \int_{1}^{\infty} \frac{1}{x^p} \, d(-\cos x) = -\frac{\cos x}{x^p} \bigg|_1^\infty - p \cdot \int_{1}^{\infty} \frac{\cos x}{x^{p+1}} \, dx.
	\]
	Двойная подстановка стремится к нулю, а интеграл \(\displaystyle \int_{1}^{\infty} \frac{\cos x}{x^{p+1}} \, dx\) абсолютно сходится: так как \(\left|\dfrac{\cos x}{x^{p+1}} \right| \leqslant \left|\dfrac{1}{x^{p+1}} \right|\), по \hyperlink{priz}{\bfseries признаку сравнения} при \(p > 0\) из сходимости второго следует сходимость первого. Значит, рассматриваемый интеграл сходится, но об абсолютной сходимости говорить нельзя, так как его мы рассматривали не по модулю.
	
	Теперь покажем, что при \(p \leqslant 0\) интеграл расходится. Воспользуемся \hyperlink{Критерий Больцано -- Коши сходимости несобственного интеграла}{\bfseries критерием Больцано -- Коши}: если мы предъявим такую последовательность чисел \(A_k, B_k \to +\infty\), что \(\displaystyle \int_{A_k}^{B_k} \frac{\sin x}{x^p} \not\to 0\) при \(k \to \infty\), то докажем, что интеграл расходится.
	
	Такая последовательность есть: пусть \(A_k = 2\pi k, B_k = 2\pi k + \pi\), тогда \[
		\int_{2\pi k}^{2\pi k + \pi} \frac{\sin x}{x^p} \, dx \geqslant \frac{1}{(2\pi k)^p} \int_{2\pi k}^{2\pi k + \pi} \sin x \, dx = \frac{2}{(2\pi k)^p} \not\to 0.
	\]
	
	Теперь об абсолютной сходимости. Так как \(\left|\dfrac{\sin x}{x^p}\right| \leqslant \left|\dfrac{1}{x^p}\right|\), по признаку сравнения при \(p > 1\) интеграл абсолютно сходится.
	
	Проверим, что происходит при \(p \leqslant 1\). Опять воспользуемся критерием Больцано -- Коши и докажем, что интеграл не абсолютно сходится. Пусть \(A_k = \pi k, B_k = 2 \pi k\), тогда \[
		\int_{\pi k}^{2\pi k} \left|\frac{\sin x}{x^p} \right| \, dx \geqslant \int_{\pi k}^{2\pi k} \left|\frac{\sin x}{x} \right| \, dx \geqslant \frac{1}{2\pi k} \int_{\pi k}^{2\pi k} \left|\sin x \right| \, dx = \frac{2k}{2\pi k} \not\to 0.
	\]
\end{example}

\subsection{Признак Абеля -- Дирихле сходимости несобственного интеграла}

\begin{theorem}
	Пусть функция \(f\) допустима на \([a, b)\). Обозначим \(F(A) = \int_a^A f\), где \(A \in [a, b)\). Тогда:
	\begin{description}
		\item[Признак Дирихле:] Пусть \(F(A)\) ограничена, то есть \[
			\exists C_1 > 0 \quad \forall A \in [a, b) \quad |F(A)| < C_1.
		\]
		Пусть также функция \(g\) непрерывно дифференцируема на \([a, b)\), \(g(x) \xrightarrow[x \to b-0]{} 0\), \(g(x)\) монотонна. Тогда \(\int_a^b fg\) сходится.
		\item[Признак Абеля:] Пусть \(\int_a^b f\) сходится. Пусть также \(g\) непрерывно дифференцируема на \([a, b)\), ограничена и монотонна, то есть \[
			\exists C_2 > 0 \quad \forall x \in [a, b) \quad |g(x)| < C_2.
		\]
		Тогда \(\int_a^b fg\) сходится. 
	\end{description} 
\end{theorem}
\begin{proof}
	Докажем признаки:
	\begin{description}
		\item[Признак Дирихле] Зафиксируем \(B\) и проинтегрируем \(\displaystyle \int_a^B f(x)g(x) \, dx\) по частям (напомним, что по \hyperlink{t8}{\bfseries теореме Барроу} \(F'(x) = f(x)\)):
		\begin{multline*}
			\int_a^B f(x)g(x) \, dx = \int_a^B g(x) \, d(F) = F(x)g(x) \bigg|_a^B -\int_a^B F(x)g'(x) \, dx = \\
			= F(B)g(B) - F(a)g(a) - \int_a^B F(x)g'(x) \, dx.
		\end{multline*}
		Теперь устремим \(B\) к \(b\) и посмотрим, как поведёт себя выражение: двойная подстановка имеет предел, так как в силу того, что по условию \(F(A)\) ограничена, а \(g(x) \xrightarrow[x \to b-0]{} 0\), выражение \(F(B)g(B)\) стремится к нулю. Разберёмся с интегралом: в силу монотонности \(g(x)\) её производная \(g'(x)\) постоянного знака. Но тогда наш интеграл сходится, и, более того, абсолютно сходится: \[
			\int_a^B |F(x)g'(x)| \, dx \leqslant C_1 \int_a^B |g'(x)| \, dx = \pm C_1 \int_a^B g'(x) \, dx = g(x) \bigg|_a^B.
		\]
		Двойная подстановка имеет предел, так как \(g(x) \xrightarrow[x \to b-0]{} 0\).
		\item[Признак Абеля] Пусть \(\lim\limits_{x \to b-0} g(x) = l\). Так как \(g(x)\) ограничена и монотонна, он существует и конечен. Тогда рассмотрим: \[
			\int_a^b fg = \int_a^b f(g - l) + \int_a^b f \cdot l.
		\]
		Тогда наш интеграл сходится, так как \(\int_a^b f \cdot l\) сходится по условию, а также легко проверить, что \(\int_a^b f(g - l)\) сходится по признаку Дирихле.
	\end{description}
\end{proof}

\subsection{Интеграл Дирихле}

\begin{theorem}
	\[
		\int_{0}^{+\infty} \frac{\sin x}{x} \, dx = \frac{\pi}{2}.
	\]
\end{theorem}
\begin{proof}
	Рассмотрим сумму \(\cos x + \cos 2x + \ldots + \cos nx\). Домножим на \(2 \sin \frac{x}{2}\) и, воспользовавшись формулой произведения синуса на косинус \(\sin \alpha \cdot \cos \beta = \frac{1}{2} (\sin(\alpha - \beta) + \sin(\alpha + \beta))\), получим телескопическую сумму. Обратно поделим всё на \(2 \sin \frac{x}{2}\) и получим \[
		\cos x + \cos 2x + \ldots + \cos nx = \frac{\sin \left(n + \frac{1}{2} \right)x}{2 \sin \frac{x}{2}} - \frac{1}{2}.
	\]
	Проинтегрируем это равенство на промежутке  \([0, \pi]\): \[
		0 = \int_{0}^{\pi} \frac{\sin \left(n + \frac{1}{2} \right)x}{2 \sin \frac{x}{2}} \, dx - \frac{\pi}{2}.
	\]
	Иными словами, \[
		\int_{0}^{\pi} \frac{\sin \left(n + \frac{1}{2} \right)x}{2 \sin \frac{x}{2}} \, dx = \frac{\pi}{2}.
	\]
	Теперь получше приглядимся к такому интегралу: \[
		\int_{0}^{\pi} \frac{\sin \left(n + \frac{1}{2} \right)x}{x} \, dx \xlongequal[y = \left(n + \frac{1}{2} \right)x]{} \int_{0}^{\left(n + \frac{1}{2} \right)\pi} \frac{\sin y}{y} \, dy.
	\]
	Правая часть при  \(n \to \infty\) --- это как раз интеграл Дирихле.
	Теперь, если мы докажем, что разность
	\begin{multline*}
		\int_{0}^{\pi} \frac{\sin \left(n + \frac{1}{2} \right)x}{2 \sin \frac{x}{2}} \, dx - \int_{0}^{\pi} \frac{\sin \left(n + \frac{1}{2} \right)x}{x} \, dx = \\
		= \int_{0}^{\pi} \frac{\sin \left(n + \frac{1}{2} \right)x}{2 \sin \frac{x}{2}} - \frac{\sin \left(n + \frac{1}{2} \right)x}{x} \, dx = \\
		= \int_{0}^{\pi} \sin \left(n + \tfrac{1}{2} \right)x \cdot \left(\frac{1}{2 \sin \frac{x}{2}} - \frac{1}{x}\right) \, dx
	\end{multline*}
	стремится к нулю при \(n \to \infty\), мы докажем теорему.
	
	Сделаем наблюдение: 
	\begin{multline*}
		\int_{0}^{\pi} \sin Nx \cdot f(x) \, dx = \int_{0}^{\pi} f(x) \, d\left(-\frac{\cos Nx}{N} \right) = \\
		= -\frac{\cos Nx}{N} \cdot f(x) \bigg|_{0}^{\pi}  - \frac{1}{N} \int_{0}^{\pi} \cos Nx \cdot f'(x) \, dx \xrightarrow[N \to \infty]{} 0.
	\end{multline*}
	Условный \(\sin Nx\) у нас имеется, теперь, если мы докажем, что то, что у нас играет роль \(f(x)\), подходит для интегрирования по частям, то есть непрерывно дифференцируемо, то искупаемся в шоколадi.
	
	К непрерывности вопросов не возникает нигде, кроме окрестности нуля. Однако и там функция непрерывна --- это легко видеть, если привести к общему знаменателю и заменить в нём \(\sin \frac{x}{2}\) на эквивалентную \(\frac{x}{2}\), а в числителе разложить его в ряд Тейлора до второго порядка.
	
	Осталось показать, что производная тоже непрерывна (к ней вопросы также только в окрестности нуля): для этого рассмотрим предел \(\lim\limits_{x \to 0} f'(x)\) --- по следствию из теоремы Лагранжа, если он существует, то равен как раз значению производной в нуле. Чтобы показать, что он существует, опять приведём к общему знаменателю, заменим все синусы и косинусы на эквивалентные и получим в числителе и знаменателе выражения порядка \(x^4\).
	
	Получается, что наша функция непрерывно дифференцируема, а значит, теорема доказана.
\end{proof}

\subsection{Две леммы об интегрировании асимптотических равенств}

\begin{theorem}
	ДОДЕЛАТЬ
\end{theorem}
\begin{proof}
	ДОДЕЛАТЬ
\end{proof}

\subsection{Иррациональность \(e^2\)}

\begin{theorem}
	ДОДЕЛАТЬ
\end{theorem}
\begin{proof}
	ДОДЕЛАТЬ
\end{proof}

\subsection{Свойства рядов: линейность, свойства остатка, необходимое условие сходимости, критерий Больцано~--~Коши}

\begin{theorem} \hypertarget{свойства рядов}{}
	Некоторые свойства рядов:
	\begin{enumerate}
		\item Пусть ряды  \(\sum a_k, \sum b_k\) сходятся. Тогда ряд \(\sum c_k\), где \(c_k = a_k + b_k\), тоже сходится, причём \(\sum c_k = \sum a_k + \sum b_k\).
		\item Пусть ряд \(\sum a_k\) сходится. Тогда для любого \(\lambda \in \mathbb{R}\) ряд \(\sum \lambda a_k\) тоже сходится, причём \(\sum \lambda a_k = \lambda \sum a_k\).
		\item Если ряд \(\sum a_k\) сходится, то любой его остаток \(R_N\) тоже сходится.
		\item Если какой-то остаток \(R_N\) ряда \(\sum a_n\) сходится, то и сам ряд сходится.
		\item Ряд \(\sum a_k\) сходится тогда и только тогда, когда \(R_N \to 0\) при \(N \to \infty\).
	\end{enumerate}
\end{theorem}
\begin{proof}
	Ну-с:
	\begin{enumerate}
		\item Очевидно, что для любого \(N\) \(S_N^c = S_N^a + S_N^b\). Делаем предельный переход при \(N \to \infty\) --- предел левой части существует, так как существует предел правой.
		\item Опять-таки очевидно, что для любого \(N\) \(S_N^{\lambda a} = \lambda S_N^a\). Делаем предельный переход, и всё.
		\item Пусть \(n > m\). Рассмотрим \[
		\sum_{k=1}^{n} a_k = \sum_{k=1}^{m-1} a_k + \sum_{k=m}^{n} a_k
		\]
		и перейдём к пределу при \(n \to \infty\): \[
		\sum_{k=1}^{+\infty} a_k = \sum_{k=1}^{m} a_k + \sum_{k=m+1}^{+\infty} a_k.
		\]
		Тогда из существования предела левой части следует существование предела правой, а из существования предела правой --- существование предела левой.
		\item Очевидно из доказательства предыдущего пункта.
		\item Рассмотрим \[
		\sum_{k=1}^{+\infty} a_k = \sum_{k=1}^{N-1} a_k + R_N.
		\]
		Допустим, ряд сходится. Тогда выражение \(\sum\limits_{k=1}^{N-1} a_k\) стремится к сумме ряда при \(N \to \infty\), а значит, \(R_N\) стремится к нулю.
		
		В обратную сторону доказывать вообще ничего не нужно: мы уже доказали, что если \(R_N\) сходится, то и ряд тоже.
	\end{enumerate}
\end{proof}

\hypertarget{необходимое условие сходимости}{}
\begin{corollary}[необходимое условие сходимости]
	Если ряд \(\sum a_k\) сходится, то \(a_k \to 0\) при \(k \to \infty\).
\end{corollary}
\begin{proof}
	Очевидно из того, что \(a_N = R_N - R_{N+1}\).
\end{proof}

\begin{theorem}[критерий Больцано -- Коши]
	Ряд \(\sum a_k\) сходится тогда и только тогда, когда \[
		\forall \varepsilon > 0 \quad \exists N \quad \forall n > N \quad \forall p \in \mathbb{N} \quad |a_{n+1} + a_{n+2} + \ldots + a_{n+p}| < \varepsilon.
	\]
\end{theorem}
\begin{proof}
	В силу того, что \(|a_{n+1} + a_{n+2} + \ldots + a_{n+p}| = |S_{n + p} - S_n|\), доказывать тут, строго говоря, нечего, так как мы просто записали определение существования предела частичных сумм.
\end{proof}

\subsection{\itshape Признак сравнения сходимости положительных рядов}

\begin{lemma}
	Пусть \(a_n \geqslant 0\). Тогда следующие утверждения эквиваленты:
	\begin{enumerate}
		\item Ряд \((A)\) сходится,
		\item Последовательность частичных сумм \((S_n)\) ограничена.
	\end{enumerate}
\end{lemma}
\begin{proof}
	В силу того, что \((S_n)\) монотонна, по теореме о пределе монотонной функции существование предела последовательности (а это и означает сходимость ряда) эквивалентна её ограниченности.
\end{proof}

\begin{theorem} \hypertarget{признак сравнения рядов}{}
	Пусть \((A), (B)\) --- положительные ряды. Тогда справедливы следующие утверждения:
	\begin{enumerate}
		\item Пусть для любого \(n\) \(a_n \leqslant b_n\). Тогда:
		\begin{enumerate}
			\item \((B)\) сходится \(\Rightarrow\) \((A)\) сходится,
			\item \((A)\) расходится \(\Rightarrow\) \((B)\) расходится.
		\end{enumerate}
		\item Пусть \(\lim\limits_{k \to \infty} \frac{a_k}{b_k} = l \in [0, +\infty]\). Тогда:
		\begin{enumerate}
			\item Если \(l = +\infty\), то:
			\begin{enumerate}
				\item \((A)\) сходится \(\Rightarrow\) \((B)\) сходится,
				\item \((B)\) расходится \(\Rightarrow\) \((A)\) расходится.
			\end{enumerate}
			\item Если \(l = 0\), то:
			\begin{enumerate}
				\item \((B)\) сходится \(\Rightarrow\) \((A)\) сходится,
				\item \((A)\) расходится \(\Rightarrow\) \((B)\) расходится.
			\end{enumerate}
			\item Если \(l \in (0, +\infty)\), то \((A)\) и \((B)\) сходятся и расходятся одновременно.
		\end{enumerate}
	\end{enumerate}
\end{theorem}
\begin{proof}
	Можно почти что скопировать доказательство \hyperlink{priz}{аналогичной теоремы об интегралах}:
	\begin{enumerate}
		\item Очевидно из леммы.
		\item
		\begin{enumerate}
			\item Если \(\lim\limits_{k \to \infty} \dfrac{a_k}{b_k} = +\infty\), то, НСНМ, \(b_k \leqslant a_k\). Отсылаем к пункту~1.
			\item Если \(\lim\limits_{k \to \infty} \dfrac{a_k}{b_k} = 0\), то, НСНМ, \(a_k \leqslant b_k\). Опять отсылаем к пункту~1.
			\item Существует такое \(N\), что для всех \(n > N\) выполняется, например, \[
				\frac{1}{2} \, l < \frac{a_n}{b_n} < \frac{3}{2} \, l,
			\]
			то есть \[
				\frac{1}{2} \, l  \cdot b_n < a_n < \frac{3}{2} \, l \cdot b_n.
			\]
			Учитывая, что из сходимости остатка следует сходимость ряда, по пункту~1 из правой части получаем, что если \((B)\) сходится, то и \((A)\) сходится, а из левой --- что если \((A)\) сходится, то и \((B)\) сходится. Получили равносильность.
		\end{enumerate}
	\end{enumerate}
\end{proof}

\subsection{\itshape Признак Коши сходимости положительных рядов (noob)} \hypertarget{Коши-нуб}{}

\begin{theorem}
	Пусть \((A)\) --- положительный ряд. Обозначим \(K_n = \sqrt[n]{a_n}\). Тогда:
	\begin{enumerate}
		\item Если существует \(q < 1\), такое, что, НСНМ, \(K_n \leqslant q\), то ряд \((A)\) сходится.
		\item Если \(K_n \geqslant 1\) для бесконечного числа номеров \(n\), то ряд \((A)\) расходится.
	\end{enumerate}
\end{theorem}
\begin{proof}
	Доказательство:
	\begin{enumerate}
		\item Пусть  \(K_n \leqslant q\), то есть \(a_n \leqslant q^n\). Ряд \(\sum q^n\) сходится как эталонный, а значит, и ряд \((A)\) также сходится.
		\item Пусть \(K_n \geqslant 1\) для бесконечного числа номеров \(n\). Это значит, что \(a_n \geqslant 1\) также для бесконечного числа номеров \(n\), а это означает, что \(a_n \not\to 0\), что противоречит \hyperlink{необходимое условие сходимости}{\bfseries необходимому условию сходимости}.
	\end{enumerate}
\end{proof}
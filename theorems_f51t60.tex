\subsection{Признак Коши сходимости положительных рядов (pro)} \hypertarget{Коши-про}{}

\begin{theorem}
	Пусть \((A)\) --- положительный ряд. Обозначим \(K_n = \sqrt[n]{a_n}\), \(K = \varlimsup\limits_{n \to \infty} K_n\). Тогда:
	\begin{enumerate}
		\item Если \(K < 1\), то ряд \((A)\) сходится,
		\item Если \(K > 1\), то ряд \((A)\) расходится.
	\end{enumerate}
\end{theorem}
\begin{proof}
	Доказательство (pro):
	\begin{enumerate}
		\item Воспользуемся \hyperlink{техническое описание верхнего предела}{\bfseries техническим описанием верхнего предела}: имеем \[
			\forall \varepsilon > 0 \quad \exists N \quad \forall n > N \quad K_n < K + \varepsilon.
		\]
		Возьмём какое-нибудь \(q \in (K, 1)\) и назначим \(\varepsilon = q - K\). Тогда, НСНМ, \(K_n < K + \varepsilon\), то есть \(K_n < q\). Согласно пункту~1 \hyperlink{Коши-нуб}{\bfseries версии признака Коши для нубов} ряд \((A)\) сходится.
		\item Опять воспользуемся техническим описанием верхнего предела: имеем \[
			\forall \varepsilon > 0 \quad \exists \textit{бесконечно много} \ n \quad K - \varepsilon < K_n.
		\]
		Возьмём \(\varepsilon = K - 1\). Тогда для бесконечного числа номеров \(n\) справедливо \(K - \varepsilon < K_n\), то есть \(1 < K_n\). Согласно уже пункту~2 версии признака Коши для нубов ряд \((A)\) расходится.
	\end{enumerate}
\end{proof}

\begin{remark}
	Если оказалось, что \(K = 1\), то признак не работает, так как можно привести примеры и сходящегося, и расходящегося рядов, для которых \(K = 1\): это, например, ряды \(\sum \frac{1}{n^2}\) и \(\sum \frac{1}{n}\) соответственно.
\end{remark}

\subsection{Признак Даламбера сходимости положительных рядов}

\begin{theorem}[noob] \hypertarget{Даламбер-нуб}{}
	Пусть \((A)\) --- положительный ряд. Обозначим \linebreak \(D_n = \frac{a_{n+1}}{a_n}\). Тогда:
	\begin{enumerate}
		\item Если существует \(q < 1\), такое, что, НСНМ, \(D_n < 1\), то ряд \((A)\) сходится,
		\item Если, НСНМ, \(D_n \geqslant 1\), то ряд \((A)\) расходится. 
	\end{enumerate}
\end{theorem}
\begin{proof}
	Докажем:
	\begin{enumerate}
		\item Имеем: \[
			\exists N \quad \forall n > N \quad \frac{a_{n+1}}{a_n} < q.
		\]
		Зафиксируем какое-нибудь число \(N + k\) и запишем:
		\begin{align*}
			\frac{a_{N+1}}{a_N} 	  &< q, 		   \\
			\frac{a_{N+2}}{a_{N+1}}   &< q, 		   \\
									  & \enskip \vdots \\
			\frac{a_{N+k}}{a_{N+k-1}} &<q. 			   \\
		\end{align*}
		Перемножим все эти неравенства: сократится всё, кроме знаменателя первой дроби и числителя последней. Получим: \[
			\frac{a_{N+k}}{a_N} < q^k,
		\]
		или \[
			a_{N+k} < q^k a_N.
		\]
		Ряд \(\sum q^k a_N\) сходится как эталонный: \(q < 1\), \(a_N\) --- просто константа. Значит, и остаток \(\sum a_{N+k}\), а значит, и ряд \((A)\), тоже сходятся.
		\item Имеем, что, НСНМ, \(a_{n+1} \geqslant a_n\). Отсюда \(a_n \not\to 0\), что противоречит \hyperlink{необходимое условие сходимости}{\bfseries необходимому условию сходимости}.
	\end{enumerate}
\end{proof}

\begin{theorem}[pro]
	Пусть \((A)\) --- положительный ряд. Обозначим \linebreak \(D_n = \frac{a_{n+1}}{a_n}\). Пусть также существует предел \(\lim\limits_{n \to \infty} D_n\). Тогда:
	\begin{enumerate}
		\item Если \(D > 1\), то ряд \((A)\) сходится.
		\item Если \(D < 1\), то ряд \((A)\) расходится.
	\end{enumerate}
\end{theorem}
\begin{proof}
	Докажем (pro):
	\begin{enumerate}
		\item Возьмём какое-нибудь \(q \in (D, 1)\). Тогда по определению предела \(D\) имеем: \[
			\exists N \quad \forall n > N \quad D_n < q.
		\]
		Следовательно, по пункту~1 \hyperlink{Даламбер-нуб}{версии признака для нубов} ряд \((A)\) сходится.
		\item По определению предела  \(D\) \[
			\exists N \quad \forall n > N \quad D_n > 1.
		\]
		Иными словами, НСНМ,  \(a_{n+1} > a_n\), откуда по пункту~2 версии признака для нубов следует расходимость ряда \((A)\).
	\end{enumerate}
\end{proof}

\begin{remark}
	Аналогично признаку Коши, если оказалось, что \(D = 1\), то признак не работает, так как можно привести примеры и сходящегося, и расходящегося рядов, для которых \(K = 1\): это, например, всё те же ряды \(\sum \frac{1}{n^2}\) и \(\sum \frac{1}{n}\) соответственно.
\end{remark}

\subsection{Признак Раабе сходимости положительных рядов}

\begin{lemma}
	Пусть \((A), (B)\) --- строго положительные ряды, причём, \linebreak НСНМ, \(\dfrac{a_{n+1}}{a_n} < \dfrac{b_{n+1}}{b_n}\). Тогда:
	\begin{enumerate}
		\item \((B)\) сходится \(\Rightarrow\) \((A)\) сходится,
		\item \((A)\) расходится \(\Rightarrow\) \((B)\) расходится.
	\end{enumerate}
\end{lemma}
\begin{proof}
	Доказательство похоже на доказательство признака Даламбера. Имеем: \[
	\exists N \quad \forall n > N \quad \frac{a_{n+1}}{a_n} < \frac{b_{n+1}}{b_n}.
	\]
	Зафиксируем какое-нибудь число \(N + k\) и запишем:
	\begin{align*}
		\frac{a_{N+1}}{a_N} 	  &< \frac{b_{N+1}}{b_N},		\\
		\frac{a_{N+2}}{a_{N+1}}   &< \frac{b_{N+2}}{b_{N+1}},	\\
								  & \enskip \vdots 	  			\\
		\frac{a_{N+k}}{a_{N+k-1}} &< \frac{b_{N+k}}{b_{N+k-1}}. \\
	\end{align*}
	Перемножим эти неравенства: \[
		\frac{a_{N+k}}{a_N} < \frac{b_{N+k}}{b_N},
	\]
	или \[
		a_{N+k} < b_{N+k} \, \frac{a_N}{b_N}.
	\]
	Так как \(\dfrac{a_N}{b_N}\) --- просто константа, по \hyperlink{priz}{признаку сравнения} из сходимости  ряда \((B)\) следует сходимость \((A)\), а из расходимости ряда \((A)\) --- расходимость  \((B)\).
\end{proof}

\begin{theorem}[noob]
	Пусть \((A)\) --- строго положительный ряд. Тогда справедливы следующие утверждения:
	\begin{enumerate}
		\item Если, НСНМ, \(n \left(\dfrac{a_n}{a_{n+1}} - 1 \right) \geqslant r > 1\), то ряд \((A)\) сходится,
		\item Если, НСНМ, \(n \left(\dfrac{a_n}{a_{n+1}} - 1 \right) \leqslant 1\), то ряд \((A)\) расходится.
	\end{enumerate}
\end{theorem}
\begin{proof}
	Сделаем наблюдение, а именно рассмотрим ряд \(\sum \frac{1}{n^p}\) и вычислим для него величину \(n \left(\dfrac{a_n}{a_{n+1}} - 1 \right)\): \[
	n \left(\left(\frac{n + 1}{n} \right)^p - 1 \right) = n \left(\left(1 + \frac{1}{n}\right)^p - 1 \right) \xrightarrow[n \to \infty]{} p.
	\]
	Теперь используем его для доказательства наших утверждений:
	\begin{enumerate}
		\item Имеем \(n \left(\dfrac{a_n}{a_{n+1}} - 1 \right) \geqslant \dfrac{r}{n}\), то есть \(\dfrac{a_n}{a_{n+1}} \geqslant 1 + \dfrac{r}{n}\). Воспользуемся наблюдением: возьмём \(p \in (1, r)\), тогда, НСНМ, \(n \left(\left(1 + \frac{1}{n}\right)^p - 1 \right) < r\), то есть \[
			\left(1 + \frac{1}{n}\right)^p < 1 + \frac{r}{n} \leqslant \dfrac{a_n}{a_{n+1}}.
		\]
		Иными словами, мы получили, что  \[
			\frac{1/n^p}{1/(n + 1)^p} < \frac{a_n}{a_{n+1}},
		\]
		что равносильно \[
			\frac{1/(n + 1)^p}{1/n^p} > \frac{a_{n+1}}{a_n}.
		\]
		Отсюда по лемме из сходимости ряда \(\sum \frac{1}{n^p}\) (что выполняется при \(p > 1\)) следует сходимость \((A)\).
		\item Теперь имеем \(\dfrac{a_n}{a_{n+1}} \leqslant 1 + \dfrac{1}{n} = \dfrac{n + 1}{n}\), или \(\dfrac{a_{n+1}}{a_n} \geqslant \dfrac{n}{n + 1}\), что равносильно \[
			\frac{a_{n+1}}{a_n} \geqslant \frac{1/(n + 1)}{1/n}.
		\]
		Отсюда по лемме из расходимости ряда \(\sum \frac{1}{n}\) следует расходимость \((A)\).
	\end{enumerate}
\end{proof}

\begin{corollary}[pro-версия]
	Пусть \((A)\) --- строго положительный ряд. Пусть также существует предел \(\lim\limits_{n \to \infty} n \left(\dfrac{a_n}{a_{n+1}} - 1 \right) = R\). Тогда:
	\begin{enumerate}
		\item Если \(R > 1\), то ряд \((A)\) сходится,
		\item Если \(R < 1\), то ряд \((A)\) расходится.
	\end{enumerate}
\end{corollary}
\begin{proof}
	Упражнение... (МОЖЕТ БЫТЬ,  ДОДЕЛАТЬ)
\end{proof}

\begin{remark}
	Аналогично двум предыдущим признакам, если оказалось, что \(R = 1\), то признак не работает, так как можно привести примеры и сходящегося, и расходящегося рядов, для которых \(R = 1\): это, например, ряды \(\sum \frac{1}{n \ln^2 n}\) и \(\sum \frac{1}{n \ln n}\) соответственно (сходимость первого и расходимость второго доказываются с помощью \hyperlink{интегральный признак Коши}{\bfseries интегрального признака Коши}).
\end{remark}

\subsection{Интегральный признак Коши сходимости числовых рядов}

\begin{theorem} \hypertarget{интегральный признак Коши}{}
	Пусть функция \(f \colon [1, +\infty) \to [0, \infty)\) непрерывна и монотонна. Тогда ряд \(\displaystyle \sum_{k=1}^{+\infty} f(k)\) и интеграл \(\displaystyle \int_{1}^{+\infty} f(x) \, dx\) сходятся и расходятся одновременно.
\end{theorem}
\begin{proof}
	ДОДЕЛАТЬ РИСУНОК \\
	На рисунке показано, что \[
		\left|\sum_{k=1}^{+\infty} f(k) - \int_{1}^{+\infty} f(x) \, dx \right| \leqslant |f(1) - f(n)|.
	\]
	Это значит, что
	\begin{equation} \label{intcauchy}
		S_n = \displaystyle \int_{1}^{+\infty} f(x) \, dx + \Delta_n, \quad \text{где} \ |\Delta_n| \leqslant |f(1) - f(n)|.
	\end{equation}
	
	Случай \(f(x) \to +\infty\) тривиален, другие же случае предполагают, что функция \(f\) ограничена.
	
	Из рисунка очевидно, что величина \(\Delta_n\) с ростом \(n\) увеличивается по модулю, а также что она постоянного знака: в случае убывающей функции отрицательная, в случае возрастающей --- положительная. Из её ограниченности следует, что \(\Delta_n\) имеет предел при  \(n \to \infty\).
	
	А это значит, что пределы левой и правой частей равенства~\eqref{intcauchy} существуют или не существуют одновременно.
	
	МБ ДОДЕЛАТЬ ФОРМАЛЬНО
\end{proof}

\subsection{\itshape Признак Лейбница}
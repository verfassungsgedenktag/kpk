\subsection{Признак Коши сходимости положительных рядов (pro)} \hypertarget{Коши-про}{}

\begin{theorem}
	Пусть \((A)\) --- положительный ряд. Обозначим \(K_n = \sqrt[n]{a_n}\), \(K = \varlimsup\limits_{n \to \infty} K_n\). Тогда:
	\begin{enumerate}
		\item Если \(K < 1\), то ряд \((A)\) сходится,
		\item Если \(K > 1\), то ряд \((A)\) расходится.
	\end{enumerate}
\end{theorem}
\begin{proof}
	Доказательство (pro):
	\begin{enumerate}
		\item Воспользуемся \hyperlink{техническое описание верхнего предела}{\bfseries техническим описанием верхнего предела}: имеем \[
			\forall \varepsilon > 0 \quad \exists N \quad \forall n > N \quad K_n < K + \varepsilon.
		\]
		Возьмём какое-нибудь \(q \in (K, 1)\) и назначим \(\varepsilon = q - K\). Тогда, НСНМ, \(K_n < K + \varepsilon\), то есть \(K_n < q\). Согласно пункту~1 \hyperlink{Коши-нуб}{\bfseries версии признака Коши для нубов} ряд \((A)\) сходится.
		\item Опять воспользуемся техническим описанием верхнего предела: имеем \[
			\forall \varepsilon > 0 \quad \exists \textit{бесконечно много} \ n \quad K - \varepsilon < K_n.
		\]
		Возьмём \(\varepsilon = K - 1\). Тогда для бесконечного числа номеров \(n\) справедливо \(K - \varepsilon < K_n\), то есть \(1 < K_n\). Согласно уже пункту~2 версии признака Коши для нубов ряд \((A)\) расходится.
	\end{enumerate}
\end{proof}

\begin{remark}
	Если оказалось, что \(K = 1\), то признак не работает, так как можно привести примеры и сходящегося, и расходящегося рядов, для которых \(K = 1\): это, например, ряды \(\sum \frac{1}{n^2}\) и \(\sum \frac{1}{n}\) соответственно.
\end{remark}

\subsection{Признак Даламбера сходимости положительных рядов}

\begin{theorem}[noob] \hypertarget{Даламбер-нуб}{}
	Пусть \((A)\) --- положительный ряд. Обозначим \linebreak \(D_n = \frac{a_{n+1}}{a_n}\). Тогда:
	\begin{enumerate}
		\item Если существует \(q < 1\), такое, что, НСНМ, \(D_n < q\), то ряд \((A)\) сходится,
		\item Если, НСНМ, \(D_n \geqslant 1\), то ряд \((A)\) расходится. 
	\end{enumerate}
\end{theorem}
\begin{proof}
	Докажем:
	\begin{enumerate}
		\item Имеем: \[
			\exists N \quad \forall n > N \quad \frac{a_{n+1}}{a_n} < q.
		\]
		Зафиксируем какое-нибудь число \(N + k\) и запишем:
		\begin{align*}
			\frac{a_{N+1}}{a_N} 	  &< q, 		   \\
			\frac{a_{N+2}}{a_{N+1}}   &< q, 		   \\
									  & \enskip \vdots \\
			\frac{a_{N+k}}{a_{N+k-1}} &<q. 			   \\
		\end{align*}
		Перемножим все эти неравенства: сократится всё, кроме знаменателя первой дроби и числителя последней. Получим: \[
			\frac{a_{N+k}}{a_N} < q^k,
		\]
		или \[
			a_{N+k} < q^k a_N.
		\]
		Ряд \(\sum q^k a_N\) сходится как эталонный: \(q < 1\), \(a_N\) --- просто константа. Значит, и остаток \(\sum a_{N+k}\), а значит, и ряд \((A)\), тоже сходятся.
		\item Имеем, что, НСНМ, \(a_{n+1} \geqslant a_n\). Отсюда \(a_n \not\to 0\), что противоречит \hyperlink{необходимое условие сходимости}{\bfseries необходимому условию сходимости}.
	\end{enumerate}
\end{proof}

\begin{theorem}[pro]
	Пусть \((A)\) --- положительный ряд. Обозначим \linebreak \(D_n = \frac{a_{n+1}}{a_n}\). Пусть также существует предел \(\lim\limits_{n \to \infty} D_n = D\). Тогда:
	\begin{enumerate}
		\item Если \(D < 1\), то ряд \((A)\) сходится.
		\item Если \(D > 1\), то ряд \((A)\) расходится.
	\end{enumerate}
\end{theorem}
\begin{proof}
	Докажем (pro):
	\begin{enumerate}
		\item Возьмём какое-нибудь \(q \in (D, 1)\). Тогда по определению предела \(D\) имеем: \[
			\exists N \quad \forall n > N \quad D_n < q.
		\]
		Следовательно, по пункту~1 \hyperlink{Даламбер-нуб}{версии признака для нубов} ряд \((A)\) сходится.
		\item По определению предела  \(D\) \[
			\exists N \quad \forall n > N \quad D_n > 1.
		\]
		Иными словами, НСНМ,  \(a_{n+1} > a_n\), откуда по пункту~2 версии признака для нубов следует расходимость ряда \((A)\).
	\end{enumerate}
\end{proof}

\begin{remark}
	Аналогично признаку Коши, если оказалось, что \(D = 1\), то признак не работает, так как можно привести примеры и сходящегося, и расходящегося рядов, для которых \(K = 1\): это, например, всё те же ряды \(\sum \frac{1}{n^2}\) и \(\sum \frac{1}{n}\) соответственно.
\end{remark}

\subsection{Признак Раабе сходимости положительных рядов}

\begin{lemma}
	Пусть \((A), (B)\) --- строго положительные ряды, причём, \linebreak НСНМ, \(\dfrac{a_{n+1}}{a_n} < \dfrac{b_{n+1}}{b_n}\). Тогда:
	\begin{enumerate}
		\item \((B)\) сходится \(\Rightarrow\) \((A)\) сходится,
		\item \((A)\) расходится \(\Rightarrow\) \((B)\) расходится.
	\end{enumerate}
\end{lemma}
\begin{proof}
	Доказательство похоже на доказательство признака Даламбера. Имеем: \[
	\exists N \quad \forall n > N \quad \frac{a_{n+1}}{a_n} < \frac{b_{n+1}}{b_n}.
	\]
	Зафиксируем какое-нибудь число \(N + k\) и запишем:
	\begin{align*}
		\frac{a_{N+1}}{a_N} 	  &< \frac{b_{N+1}}{b_N},		\\
		\frac{a_{N+2}}{a_{N+1}}   &< \frac{b_{N+2}}{b_{N+1}},	\\
								  & \enskip \vdots 	  			\\
		\frac{a_{N+k}}{a_{N+k-1}} &< \frac{b_{N+k}}{b_{N+k-1}}. \\
	\end{align*}
	Перемножим эти неравенства: \[
		\frac{a_{N+k}}{a_N} < \frac{b_{N+k}}{b_N},
	\]
	или \[
		a_{N+k} < b_{N+k} \, \frac{a_N}{b_N}.
	\]
	Так как \(\dfrac{a_N}{b_N}\) --- просто константа, по \hyperlink{priz}{признаку сравнения} из сходимости  ряда \((B)\) следует сходимость \((A)\), а из расходимости ряда \((A)\) --- расходимость  \((B)\).
\end{proof}

\begin{theorem}[noob]
	Пусть \((A)\) --- строго положительный ряд. Тогда справедливы следующие утверждения:
	\begin{enumerate}
		\item Если, НСНМ, \(n \left(\dfrac{a_n}{a_{n+1}} - 1 \right) \geqslant r > 1\), то ряд \((A)\) сходится,
		\item Если, НСНМ, \(n \left(\dfrac{a_n}{a_{n+1}} - 1 \right) \leqslant 1\), то ряд \((A)\) расходится.
	\end{enumerate}
\end{theorem}
\begin{proof}
	Сделаем наблюдение, а именно рассмотрим ряд \(\sum \frac{1}{n^p}\) и вычислим для него величину \(n \left(\dfrac{a_n}{a_{n+1}} - 1 \right)\): \[
	n \left(\left(\frac{n + 1}{n} \right)^p - 1 \right) = n \left(\left(1 + \frac{1}{n}\right)^p - 1 \right) \xrightarrow[n \to \infty]{} p.
	\]
	Теперь используем его для доказательства наших утверждений:
	\begin{enumerate}
		\item Имеем \(\left(\dfrac{a_n}{a_{n+1}} - 1 \right) \geqslant \dfrac{r}{n}\), то есть \(\dfrac{a_n}{a_{n+1}} \geqslant 1 + \dfrac{r}{n}\). Воспользуемся наблюдением: возьмём \(p \in (1, r)\), тогда, НСНМ, \(n \left(\left(1 + \frac{1}{n}\right)^p - 1 \right) < r\), то есть \[
			\left(1 + \frac{1}{n}\right)^p < 1 + \frac{r}{n} \leqslant \dfrac{a_n}{a_{n+1}}.
		\]
		Иными словами, мы получили, что  \[
			\frac{1/n^p}{1/(n + 1)^p} < \frac{a_n}{a_{n+1}},
		\]
		что равносильно \[
			\frac{1/(n + 1)^p}{1/n^p} > \frac{a_{n+1}}{a_n}.
		\]
		Отсюда по лемме из сходимости ряда \(\sum \frac{1}{n^p}\) (что выполняется при \(p > 1\)) следует сходимость \((A)\).
		\item Теперь имеем \(\dfrac{a_n}{a_{n+1}} \leqslant 1 + \dfrac{1}{n} = \dfrac{n + 1}{n}\), или \(\dfrac{a_{n+1}}{a_n} \geqslant \dfrac{n}{n + 1}\), что равносильно \[
			\frac{a_{n+1}}{a_n} \geqslant \frac{1/(n + 1)}{1/n}.
		\]
		Отсюда по лемме из расходимости ряда \(\sum \frac{1}{n}\) следует расходимость \((A)\).
	\end{enumerate}
\end{proof}

\begin{corollary}[pro-версия]
	Пусть \((A)\) --- строго положительный ряд. Пусть также существует предел \(\lim\limits_{n \to \infty} n \left(\dfrac{a_n}{a_{n+1}} - 1 \right) = R\). Тогда:
	\begin{enumerate}
		\item Если \(R > 1\), то ряд \((A)\) сходится,
		\item Если \(R < 1\), то ряд \((A)\) расходится.
	\end{enumerate}
\end{corollary}
\begin{proof}
	Упражнение... (МОЖЕТ БЫТЬ,  ДОДЕЛАТЬ)
\end{proof}

\begin{remark}
	Аналогично двум предыдущим признакам, если оказалось, что \(R = 1\), то признак не работает, так как можно привести примеры и сходящегося, и расходящегося рядов, для которых \(R = 1\): это, например, ряды \(\sum \frac{1}{n \ln^2 n}\) и \(\sum \frac{1}{n \ln n}\) соответственно (сходимость первого и расходимость второго доказываются с помощью \hyperlink{интегральный признак Коши}{\bfseries интегрального признака Коши}).
\end{remark}

\subsection{Интегральный признак Коши сходимости числовых рядов}

\begin{theorem} \hypertarget{интегральный признак Коши}{}
	Пусть функция \(f \colon [1, +\infty) \to [0, +\infty)\) непрерывна и монотонна. Тогда ряд \(\displaystyle \sum_{k=1}^{+\infty} f(k)\) и интеграл \(\displaystyle \int_{1}^{+\infty} f(x) \, dx\) сходятся и расходятся одновременно.
\end{theorem}
\begin{proof}
	ДОДЕЛАТЬ РИСУНОК \\
	На рисунке показано, что \[
		\left|\sum_{k=1}^{+\infty} f(k) - \int_{1}^{+\infty} f(x) \, dx \right| \leqslant |f(1) - f(n)|.
	\]
	Это значит, что
	\begin{equation} \label{intcauchy}
		S_n = \displaystyle \int_{1}^{+\infty} f(x) \, dx + \Delta_n, \quad \text{где} \ |\Delta_n| \leqslant |f(1) - f(n)|.
	\end{equation}
	
	Случай \(f(x) \to +\infty\) тривиален, другие же случае предполагают, что функция \(f\) ограничена.
	
	Из рисунка очевидно, что величина \(\Delta_n\) с ростом \(n\) увеличивается по модулю, а также что она постоянного знака: в случае убывающей функции отрицательная, в случае возрастающей --- положительная. Из её ограниченности следует, что \(\Delta_n\) имеет предел при  \(n \to \infty\).
	
	А это значит, что пределы левой и правой частей равенства~\eqref{intcauchy} существуют или не существуют одновременно.
	
	МБ ДОДЕЛАТЬ ФОРМАЛЬНО
\end{proof}

\subsection{\itshape Признак Лейбница}

\begin{theorem}
	Пусть \(\sum (-1)^n c_k\) --- знакопеременный ряд. Пусть также \(c_k \to 0\) при  \(k \to \infty\). Тогда ряд \(\sum (-1)^n c_k\) сходится.
\end{theorem}
\begin{proof}
	ДОДЕЛАТЬ
\end{proof}

\subsection{Признаки Дирихле и Абеля сходимости числового ряда}

\begin{theorem}
	Рассмотрим ряд вида \(\sum a_k b_k\). Обозначим \(A_n = a_1 + a_2 + \ldots + a_n\). Тогда справедливы утверждения:
	\begin{description}
		\item[Признак Дирихле.] Если последовательность \((A_n)\) ограничена, то есть \(\exists C_A > 0 \enskip \forall n \enskip |A_n| \leqslant C_A\), а последовательность \((b_k)\) монотонно стремится к нулю, то ряд \(\sum a_k b_k\) сходится.
		\item[Признак Абеля.] Если ряд \(\sum a_k\) сходится, а последовательность \((b_k)\) монотонна и ограничена, то есть  \(\exists C_b > 0 \enskip \forall k \enskip |b_k| \leqslant C_b\), то ряд \(\sum a_k b_k\) сходится.
	\end{description}
\end{theorem}
\begin{proof}
	Сделаем наблюдение (можно назвать это суммированием по частям, но вообще это \textit{преобразование Абеля}):
	\begin{equation} \label{dirab_r}
		\sum_{k=1}^N a_k b_k = A_N b_N + \sum_{k=1}^{N-1} A_k (b_k - b_{k+1})
	\end{equation}
	(справедливость можно проверить, посмотрев, что каждое \(b_k\) входит в левую и правую части с одинаковыми коэффициентами: слева это \(a_k\), справа это \(A_k - A_{k-1} = a_k\)). Применим его для доказательства признаков:
	\begin{description}
		\item[Признак Дирихле.] В равенстве~\eqref{dirab_r} перейдём к пределу при \(N \to \infty\). В силу того, что \(A_N \leqslant C_A\), а \(b_N \to 0\), выражение \(A_N b_N \to 0\).
		
		Мы также заявляем, что ряд \(\sum A_k (b_k - b_{k+1})\) абсолютно сходится. Почему? Проверим, что последовательность частичных сумм ряда  \(\sum |A_k| \cdot |b_k - b_{k+1}|\) ограничена (из этого в силу монотонности этой последовательности будет следовать существование конечного предела).
		
		Заметим, что в силу монотонности \((b_k)\) выражение \(b_k - b_{k+1}\) постоянного знака. Отсюда имеем \[
			\sum_{k=1}^{N-1} |A_k| \cdot |b_k - b_{k+1}| \leqslant C_A \sum_{k=1}^{N-1} |b_k - b_{k+1}| = \pm C_A \sum_{k=1}^{N-1} (b_k - b_{k+1}).
		\]
		Получили телескопическую сумму, равную \(\pm C_A (b_1 - b_N)\). Из того, что \(b_N \to 0\) при  \(N \to \infty\), очевидно, что последовательность \(\pm C_A (b_1 - b_N)\) ограничена. Отсюда получаем, что ряд \(\sum A_k (b_k - b_{k+1})\) абсолютно сходится, а значит, исходный ряд сходится.
		\item[Признак Абеля.] Так как последовательность \((b_k)\) монотонна и ограничена, у неё существует конечный предел \(\beta\). Запишем \[
			\sum_{k=1}^N a_k b_k = \beta \sum_{k=1}^N a_k + \sum_{k=1}^N a_k (b_k - \beta)
		\]
		(просто прибавили и вычли \(\beta\) одинаковое количество раз). При переходе к пределу при  \(N \to \infty\) выражение \(\beta \sum a_n\) имеет конечный предел.
		
		Выражение же  \(\sum a_k (b_k - \beta)\) сходится по вышедоказанному признаку Дирихле, так как из сходимости ряда \(\sum a_n\) следует ограниченность \((A_n)\), а \((b_k - \beta)\) очевидно монотонно стремится к нулю при \(k \to \infty\).
	\end{description}
\end{proof}

\subsection{Теорема о группировке слагаемых}

\begin{theorem}
	Рассмотрим ряд \((A) = (a_1 + \ldots + a_{n_1}) + (a_{n_1+1} + \ldots + a_{n_2}) + \ldots\) Обозначим \(b_k = a_{n_{k-1}+1} + \ldots + a_{n_k}\).  Тогда справедливы утверждения:
	\begin{enumerate}
		\item Если ряд \((A)\) сходится, то ряд \((B)\) сходится и имеет ту же сумму,
		\item Если \((A)\) --- положительный ряд, то \(S^a = S^b\) (или суммы рядов равны бесконечности).
	\end{enumerate}
\end{theorem}
\begin{proof}
	Тривиально: из того, что \(S_m^b = S_{n_m}^a\), в первом случае получаем, что \(S_m^b\) стремится к сумме ряда \((A)\), а во втором --- что они сходятся к одному конечному значению или к бесконечности одновременно (ситуация, в которой предела не существует, исключена, так как наши ряды положительны).
\end{proof}

МБ ДОДЕЛАТЬ ЗАМЕЧАНИЯ

\subsection{Теорема о перестановке слагаемых}

\begin{theorem} \hypertarget{теорема о перестановке слагаемых}{}
	Рассмотрим ряд \(\sum a_k\) и его перестановку \(\sum b_k\). Тогда если ряд \(\sum a_k\) абсолютно сходится, то ряд \(\sum b_k\) абсолютно сходится к той же сумме.
\end{theorem}
\begin{proof}
	Рассмотрим разные случаи:
	\begin{enumerate}
		\item Пусть \(\sum a_k\) --- положительный ряд. По определению частичная сумма \(S_k^b = a_{\omega(1)} + \ldots + a_{\omega(k)} \leqslant S_M^a\), где \(M = \max (\omega(1), \ldots, \omega(k))\). Так как \(M \to \infty\) при \(k \to \infty\), а пределы частичных сумм существуют (так как \(a_k \geqslant 0\)), получаем, что \(S^b \leqslant S^a\). А из того, что существует обратное отображение \(\omega^{-1}\), путём аналогичных рассуждений получаем, что \(S^a \leqslant S^b\). Отсюдо следует равенство.
		\item Уберём условие положительности ряда \(\sum a_k\). Рассмотрим ряды \linebreak \(\sum a_k^+, \sum b_k^+\), где \(a_k^+ = \max (a_k, 0)\), \(b_k^+ = \max (b_k, 0)\). Тогда ряд \(\sum b_k^+\) --- перестановка \(\sum a_k^+\), которая задаётся той же биекцией \(\omega\), и по пункту~1 они сходятся к одному значению. 
		
		Аналогично определим \(a_k^- = \max (-a_k, 0)\), \(b_k^- = \max (-b_k, 0)\). Эти ряды тоже являются перестановками друг друга и тоже сходятся к одному значению. А из того, что \(\sum a_k = \sum a_k^+ +\sum a_k^-\) (аналогичное, очевидно, справедливо и для \(\sum b_k\)), следует утверждение теоремы.
	\end{enumerate}
\end{proof}

\subsection{Теорема о произведении рядов}

\begin{theorem}
	Пусть ряды \(\sum a_k, \sum b_k\) абсолютно сходятся и их суммы равны \(S^a\) и \(S^b\) соответственно. Тогда для любой биекции \(\gamma \colon \mathbb{N} \to \mathbb{N} \times \mathbb{N}\), которая переводит \(x\) в \((\varphi(x), \psi(x))\), произведение рядов \(\sum a_k, \sum b_k\) --- абсолютно сходящийся ряд, сумма которого равна \(S^a \cdot S^b\).
\end{theorem}
\begin{proof}
	Обозначим \(\sum |a_k| = S_*^a, \sum |b_k| = S_*^b\) и исследуем произведение рядов на абсолютную сходимость: \[
		\sum_{k=1}^{N} |a_{\varphi(k)}| \cdot |b_{\psi(k)}| \leqslant \sum_{k=1}^{n} |a_k| \cdot \sum_{k=1}^{m} |b_k| \leqslant S_*^a \cdot S_*^b,
	\]
	где \(n = \max(\varphi(1), \ldots, \varphi(N)), \ m = \max(\psi(1), \ldots, \psi(N))\). Получаем, что множество частичных сумм нашего произведения (где каждое слагаемое взято по модулю) ограничено, а значит, оно сходится (то есть произведение сходится абсолютно).
	
	Но ведь мы можем взять вместо \(\gamma\) какую-нибудь другую биекцию \(\widetilde{\gamma}\). Очевидно, что полученное с её помощью произведение \(\sum a_{\widetilde{\varphi}(k)} b_{\widetilde{\psi}(k)}\) --- перестановка произведения \(\sum a_{\varphi(k)} b_{\psi(k)}\). А значит, оба этих произведения сходятся к той же сумме.
	
	Возьмём в качестве \(\gamma\) <<нумерацию по квадратам>>.
	
	ДОДЕЛАТЬ РИСУНОК
	
	Тогда получим \[
		\sum_{k=1}^{n^2} a_{\varphi(k)} b_{\psi(k)} = \left(\sum_{k=1}^n a_k \right) \left(\sum_{k=1}^n b_k \right) \xrightarrow[n \to \infty]{} S^a \cdot S^b.
	\]
\end{proof}

\subsection{\itshape Неравенство Йенсена для сумм, формулировка для рядов}

\begin{theorem} \hypertarget{Йенсен-суммы}{}
	Пусть функция \(f\) выпукла на \(\langle a, b \rangle\). Тогда для любых \(x_1, \ldots, x_n\) из \(\langle a, b \rangle\) и любых неотрицательных \(\alpha_1, \ldots, \alpha_2\), таких, что \(\sum \alpha_i = 1\) выполняется \[
		f \left(\sum_{k=1}^n \alpha_k x_k \right) \leqslant \sum_{k=1}^n \alpha_k f(x_k). 
	\]
\end{theorem}
\begin{proof}
	Сначала покажем, что выражения, участвующие в неравенстве, корректны, то есть что \(x^* = \sum \alpha_k x_k\) лежит в  \(\langle a, b \rangle\). Заменим все \(x_k\) на \(x_M = \max(x_1, \ldots, x_n)\), тогда \[
		x^* \leqslant (\alpha_1 + \ldots + \alpha_n) \, x_M = x_M \in \langle a, b \rangle.
	\]
	Аналогично \(x^*\) больше минимального из \(x_k\), откуда получаем, что \(x^* \in \langle a, b \rangle\).
	
	В точке \(x^*\) проведём опорную прямую \(y = px + q\) к графику функции \(f\). Тогда \[
		f(x^*) = px^* + q = p \sum_{k=1}^n \alpha_k x_k + q \sum_{k=1}^n \alpha_k = \sum_{k=1}^n \alpha_k (px_k + q).
	\]
	Так как \((px_k + q)\) --- это точка на опорной прямой с абсциссой \(x_k\), в силу выпуклости \(f\) (опорная прямая лежит не выше графика) \(px_k + q \leqslant f(x_k)\). Получаем \[
		f(x^*) = \sum_{k=1}^n \alpha_k (px_k + q) \leqslant \sum_{k=1}^n \alpha_k f(x_k).
	\]
\end{proof}

ДОДЕЛАТЬ ФОРМУЛИРОВКУ ДЛЯ РЯДОВ


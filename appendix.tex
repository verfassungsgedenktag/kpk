\section{Приложение}

\subsection{Эталонные интегралы}

\begin{lemma}
	Справедливы утверждения:
	\begin{enumerate}
		\item Интеграл \(\displaystyle \int_a^{+\infty} \frac{1}{x^p}\) сходится при \(p > 1\) и расходится в противном случае,
		\item Интеграл \(\displaystyle \int_{\to 0}^a \frac{1}{x^p}\) сходится при \(p < 1\) и расходится в противном случае.
		\item Интеграл \(\displaystyle \int_a^{+\infty} e^{-p x}\) сходится при \(p > 0\) и расходится в противном случае.
	\end{enumerate}
	
\end{lemma}
\begin{proof}
	По пунктам:
	\begin{enumerate}
		\item Имеем \[
		\int_a^{+\infty} \frac{1}{x^p} = \lim_{A \to +\infty} \int_a^A \frac{1}{x^p} = \lim_{A 	\to \infty} \left(
		\begin{cases}
			\dfrac{1}{1 - p} \cdot \dfrac{1}{x^{p - 1}}, &\text{если} \ p \neq 1, \\
			\ln x, 									   &\text{если} \ p = 1
		\end{cases}
		\right) \bigg|_a^A.
		\]
		Отсюда ясно, что при \(p - 1 > 0\) конечный предел существует, а в противном случае предел равен бесконечности.
		\item Имеем \[
		\int_{\to 0}^a \frac{1}{x^p} = \lim_{A \to 0} \int_A^a \frac{1}{x^p} = \lim_{A 	\to 0} \left(
		\begin{cases}
			\dfrac{1}{1 - p} \cdot \dfrac{1}{x^{p - 1}}, &\text{если} \ p \neq 1, \\
			\ln x, 									   &\text{если} \ p = 1
		\end{cases}
		\right) \bigg|_A^a.
		\]
		Отсюда ясно, что при \(p - 1 < 0\) конечный предел существует, а в противном случае предел равен бесконечности.
		\item Очевидно, если посмотреть на первообразную.
	\end{enumerate}
\end{proof}

\begin{remark}
	Назовём вышеописанные интегралы \textit{эталонными}.
\end{remark}

\subsection{Эталонные ряды}

\begin{lemma}
	Справедливы утверждения:
	\begin{enumerate}
		\item Ряд  \(\sum \dfrac{1}{n^p}\) сходится при \(p > 1\) и расходится в противном случае,
		\item Ряд \(\sum q^n\) сходится при \(p > 1\) и расходится в противном случае.
	\end{enumerate}
\end{lemma}
\begin{proof}
	ДОДЕЛАТЬ
\end{proof}

\begin{remark}
	Назовём вышеописанные ряды \textit{эталонными}.
\end{remark}
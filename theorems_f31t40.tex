\subsection{Формула Эйлера -- Маклорена, асимптотика степенных сумм}

\hypertarget{eumak}{}
\begin{theorem}
	Пусть \(f \in C^2 [m, n]\), где \(m, n \in \mathbb{Z}\). Тогда \[
	\int\limits_m^n f(x) dx = \sum\limits_{i = m}^n f(i) - \frac{1}{2} \int\limits_m^n f''(x) \{x\} (1 - \{x\}) dx,
	\]
	\textbf{причём} первое и последнее слагаемые в сумме \(\sum\limits_{i = m}^n f(i)\) входят в неё с весом $\dfrac{1}{2}$.
\end{theorem}

\begin{proof}
	В процессе доказательства \hyperlink{trap}{формулы трапеций} мы установили (сразу перед итоговой оценкой через \(\delta\)), что \[
	\int\limits_a^b f(x) dx - \sum_{i = 1}^n \frac{f(x_i) + f(x_{i - 1})}{2} (x_i - x_{i - 1}) = -\frac{1}{2} \int_a^b f''(x) \cdot \psi(x) dx
	\]
	(у правой части здесь опять появляется минус, который в прошлой теореме был съеден модулем). Выберем дробление на отрезки вида \([i - 1, i]\), где \(i \in (m + 1) : n\). Тогда
	\begin{gather*}
		\sum_{i = 1}^n \frac{f(x_i) + f(x_{i - 1})}{2} (x_i - x_{i - 1}) = \sum_{i = m + 1}^n \frac{f(i - 1) + f(i)}{2} = \ldots \\
		\ldots = \frac{f(m)}{2} + \frac{f(m + 1)}{2} + \frac{f(m + 1)}{2} + \ldots + \frac{f(n - 1)}{2} + \frac{f(n)}{2} = \sum\limits_{i = m}^n f(i),
	\end{gather*}
	причём первое и последнее слагаемые входят в сумму с весом \(\dfrac{1}{2}\), как нам и нужно.
	
	Теперь вспомним, что \(\psi(x) = (x_i - x) (x - x_{i - 1})\). Так как в нашем случае \(x_i\) и \(x_{i - 1}\) --- это соседние целые числа, получаем как раз \(\psi(x) = \{x\} (1 - \{x\})\). Теорема доказана, осталось только всё красиво записать.
\end{proof}

\begin{example}
	\[
	1^p + 2^p + \ldots + n^p = \frac{n^p}{2} + \frac{n^{p + 1}}{p + 1} + O(\max(1, n^{p - 1})) \enskip  \text{при} \ p > -1.
	\]
\end{example}

\begin{proof}
	Рассмотрим \(f(x) = x^p\). По \hyperlink{eumak}{формуле Эйлера -- Маклорена}\[
	1^p + 2^p + \ldots + n^p = \int_1^n x^p dx + \frac{1}{2} + \frac{n^p}{2} + \frac{p (p - 1)}{2} \int_1^n x^{p - 2} \{x\} (1 - \{x\}) dx.
	\]
	Возьмём интеграл \(x^p\): \[
	\int_1^n x^p dx = \frac{n^{p + 1} - 1}{p + 1}
	\]
	и оценим второй интеграл: \[
	0 \leqslant \int_1^n x^{p - 2} \{x\} (1 - \{x\}) dx \leqslant \frac{1}{4} \cdot \left(\frac{n^{p - 1} - 1}{p - 1} \right).
	\]
	Справедливость нижней границы очевидна, так как мы рассматриваем положительные \(x\), а справедливость верхней следует из того, что максимум функции \(\{x\} (1 - \{x\})\) достигается в точке \(x = \dfrac{1}{2}\), так как это, как говорилось выше, лишь частный случай функции \(\psi(x) = (x_i - x) (x - x_{i - 1})\), где \(x_i\) и \(x_{i - 1}\) --- соседние целые числа.
	
	Понятно, что при при \(p > 1\) правая часть превращается в \(O(n^{p - 1})\), а при \(-1 < p < 1\) --- в \(O(1)\), то наш интеграл --- \(O(\max(1, n^{p - 1}))\). Отметим, что при \(p = 1\), рассуждения заканчиваются в самом начале, так как  \(\dfrac{p (p - 1)}{2} = 0\). Закинем под \(O\) все имеющиеся константы и получим, что нужно.
\end{proof}

\subsection{Асимптотика частичных сумм гармонического ряда}

\begin{example}
	\[
	1 + \frac{1}{2} + \frac{1}{3} + \frac{1}{n} = \ln n + \gamma + o(1) \enskip \text{при} \ \gamma \in \left[\frac{4}{8}, \frac{5}{8}\right].
	\]
\end{example}

\begin{proof}
	Рассмотрим \(f(x) = \dfrac{1}{x}\). По \hyperlink{eumak}{формуле Эйлера -- Маклорена} \[
	1 + \frac{1}{2} + \frac{1}{3} + \frac{1}{n} = \int_1^n \frac{1}{x} dx + \frac{1}{2} + \frac{1}{2n} + \int_1^n \frac{1}{x^3} \{x\} (1 - \{x\}) dx.
	\] 
	Как и в прошлый раз, возьмём интеграл \(\dfrac{1}{x}\): \[
	\int_1^n \frac{1}{x} dx = \ln n.
	\]
	Заметим, что второй интеграл возрастает при увеличении \(n\) (подынтегральное выражение положительно, а промежуток интегрирование увеличивается) и оценим его: \[
	0 \leqslant \int_1^n \frac{1}{x^3} \{x\} (1 - \{x\}) dx \leqslant \frac{1}{4} \left(-\frac{1}{2n^2} + \frac{1}{2} \right) = \frac{1}{8} \left(1 - \frac{1}{n^2} \right) \leqslant \frac{1}{8}.
	\]
	Так как имеющаяся у нас $\dfrac{1}{2n}$ стремится к нулю при \(n \to \infty\), получаем: \[
	1 + \frac{1}{2} + \frac{1}{3} + \frac{1}{n} = \ln n + \gamma + o(1).
	\]
	В константу \(\gamma\) мы записали \(\dfrac{1}{2}\) и значение интеграла, то есть \(\gamma \in \left[\dfrac{4}{8}, \dfrac{5}{8}\right]\), а \(o(1)\) --- это \(\dfrac{1}{2n}\), стремящаяся к нулю при \(n \to \infty\).
\end{proof}

\begin{remark}
	Константа \(\gamma\) называется \textit{постоянной Эйлера} и равняется примерно \(0,577\ldots\)
\end{remark}

\subsection{Формула Валлиса}

\hypertarget{vallem}{}
\begin{lemma}
	Если \(n \in \mathbb{Z}_+\), то \[
	\int_0^{\frac{\pi}{2}} \sin^n x \, dx = \frac{(n - 1)!!}{n!!} \cdot
	\begin{cases}
		\dfrac{\pi}{2}, &\text{если \(n\) чётно}, \\
		1, 				&\text{если \(n\) нечётно}.
	\end{cases}
	\]
\end{lemma}

\begin{proof}
	Обозначим \(I_n = \displaystyle\int_0^{\frac{\pi}{2}} \sin^n x \, dx\). Легко проверить, что \(I_0 = \dfrac{\pi}{2}, \ I_1 = 1\). При \(n - 1 \in \mathbb{N}\) проинтегрируем по частям:
	\begin{multline*}
		I_n = \int_0^{\frac{\pi}{2}} \sin^n x \, dx = \int_0^{\frac{\pi}{2}} \sin^{n - 1} x \, d(-\cos x) = -\sin^{n - 1} x \, \cos x \bigg|_0^{\frac{\pi}{2}} + \\ 
		+ (n - 1) \int_0^\frac{\pi}{2} \sin^{n - 2} x \, \cos^2 x \, dx.
	\end{multline*}
	Учтём, что двойная подстановка обнуляется, и применим формулу \(\cos^2 x = 1 - \sin^2 x\)):
	\begin{multline*}
		I_n = (n - 1) \left(\int_0^\frac{\pi}{2} \sin^{n - 2} x \, dx - \int_0^\frac{\pi}{2} \sin^n x \, dx \right) = (n - 1) (I_{n - 2} - I_n).
	\end{multline*}
	Выражая \(I_n\), получим рекуррентное соотношение \[
	I_n = \frac{n - 1}{n} I_{n - 2}.
	\]
	Теперь, применив его несколько раз, мы можем выразить \(I_n\) через \(I_0\), если \(n\) чётно, или через \(I_1\), если \(n\) нечётно. Чётность/нечётность числителя и знаменателя будет сохраняться.
\end{proof}

\hypertarget{wall}{}
\begin{theorem}
	\[
	\pi = \lim_{n \to \infty} \frac{1}{n} \left(\frac{(2n)!!}{(2n - 1)!!} \right)^2.
	\]
\end{theorem}

\begin{proof}
	При всех \(x \in \left(0, \dfrac{\pi}{2} \right)\) выполняется неравенство \(0 < \sin x < 1\), поэтому для любого \(n \in \mathbb{N}\) \[
	\sin^{2n + 1} < \sin^{2n} < \sin^{2n - 1} x,
	\]
	а тогда и \[
	I_{2n + 1} < I_{2n} < I_{2n - 1}.
	\]
	Применяя \hyperlink{vallem}{лемму}, получаем двойное неравенство \[
	\frac{(2n)!!}{(2n + 1)!!} < \frac{(2n - 1)!!}{(2n)!!} \cdot \frac{\pi}{2} < \frac{(2n - 2)!!}{(2n - 1)!!},
	\]
	что равносильно \[
	\frac{1}{2n + 1} \left(\frac{(2n)!!}{(2n - 1)!!} \right)^2 < \frac{\pi}{2} < \frac{1}{2n} \left(\frac{(2n)!!}{(2n - 1)!!} \right)^2.
	\]
	Обозначим \(x_n = \dfrac{1}{n} \left(\dfrac{(2n)!!}{(2n - 1)!!} \right)^2\). Двойное неравенство принимает вид \[
	\frac{n}{2n + 1} x_n < \frac{\pi}{2} < \frac{1}{2} x_n.
	\]
	Из правой части получаем, что \(\pi < x_n\), из левой --- что \(x_n < \dfrac{2n + 1}{2n} \pi\). Отсюда имеем \(x_n \xrightarrow[n \to \infty]{} \pi\).
\end{proof}

\subsection{\itshape Формула Стирлинга}

\begin{theorem}
	\[
	n! \sim n^n \, e^{-n} \, \sqrt{2\pi n} \enskip \textit{при} \ n \to \infty.
	\]
\end{theorem}

\begin{proof}
	Рассмотрим \(f(x) = \ln x\). По \hyperlink{eumak}{формуле Эйлера -- Маклорена} \[
	\ln 1 + \ln 2 + \ldots + \ln n = \int_1^n \ln x \, dx + \frac{\ln n}{2} - \frac{1}{2} \int_1^n \frac{1}{x^2} \{x\} (1 - \{x\}) \, dx.
	\]
	Интегрированием по частям находим \(\displaystyle\int_1^n \ln x \, dx = n \ln n - n + 1\). Заметим, что второй интеграл возрастает с увеличением \(n\) и оценим его: \[
	0 \leqslant \int_1^n \frac{1}{x^2} \{x\} (1 - \{x\}) \, dx \leqslant \frac{1}{4} \left(1 - \frac{1}{n} \right) \leqslant \frac{1}{4}.
	\]
	Получаем \[
	\ln 1 + \ln 2 + \ldots + \ln n = n \ln n - n + \frac{\ln n}{2} + C + o(1).
	\]
	В константу \(C + o(1)\) мы записали \(1\) и значение интеграла, \(o(1)\) отражает погрешность.
	
	Так как \(\ln 1 + \ln 2 + \ldots + \ln n = \ln(n!)\), возьмём от этого всего экспоненту и получим \[
	n! = n^n e^{-n} \sqrt{n} \, e^{C + o(1)}.
	\]
	Выражение \(e^{C + o(1)}\) стремится к какой-то константе, которую мы обозначим за \(C'\). Выясним, чему она равна. Из \hyperlink{wall}{формулы Валлиса} имеем \[
	\sqrt{\pi} = \lim_{n \to \infty} \frac{1}{\sqrt{n}} \cdot \frac{(2n)!!}{(2n - 1)!!}.
	\]
	Преобразуем эту формулу: \[
	\frac{1}{\sqrt{n}} \cdot \frac{(2n)!!}{(2n - 1)!!} = \frac{1}{\sqrt{n}} \cdot \frac{2^n \cdot n!}{(2n - 1)!!} = \frac{1}{\sqrt{n}} \cdot \frac{2^n \cdot n! \cdot (2n)!!}{(2n)!} = \frac{(2^n \cdot n!)^2}{\sqrt{n} \cdot (2n)!}
	\]
	Теперь можем воспользоваться нашими наработками: \[
	\frac{(2^n \cdot n!)^2}{\sqrt{n} \cdot (2n)!} = \frac{(2^n \cdot n^n e^{-n} \sqrt{n} \, C')^2}{\sqrt{n} \cdot (2n)^{2n} e^{-2n} \sqrt{2n} \, C'} = \frac{C'}{\sqrt{2}}.
	\]
	То есть \(\sqrt{\pi} = \lim\limits_{n \to \infty} \dfrac{C'}{\sqrt{2}} = \dfrac{C'}{\sqrt{2}}\), откуда \(C' = \sqrt{2\pi}\) и \(n! = n^n e^{-n} \sqrt{2\pi n}.\) 
	
\end{proof}

\subsection{Три леммы о сверхограниченных множествах}

\begin{lemma}
	Множество \(D\) сверхограниченно в \(X\) тогда и только тогда, когда \(D\) сверхограниченно в себе.
\end{lemma}

\begin{proof}
	Достаточность очевидна, докажем необходимость.
	
	Построим множество \(E_X = \{x_0, x_1,\ldots, x_n\} \subset X\) --- конечную \(\dfrac{\varepsilon}{2}\)-сеть для \(D\). То есть имеем: \[
	\forall y \in D \quad \exists x_i \in E_X \quad \rho(x, y) < \frac{\varepsilon}{2}.
	\]
	В кажом шаре \(B_{\frac{\varepsilon}{2}} (x_i)\) выберем какое-нибудь \(y_i\). Тогда множество \linebreak \(E_Y = \{y_0, y_1,\ldots, y_m\} \subset D\), где \(m \leqslant n\), --- \(\varepsilon\)-сеть для \(D\), так как \[
	\forall y \in D \quad \exists y_i \in E_Y \quad \rho(y, y_i) \leqslant \rho(y_i, x_i) + \rho(x_i, y) < \varepsilon.
	\]
\end{proof}

\begin{lemma}
	Сверхограниченность сохраняется при равномерно непрерывном отображении.
\end{lemma}

\begin{proof}
	Пусть \(f\) --- равномерно непрерывное отображение. Запишем, что это значит: \[
	\forall \varepsilon > 0 \quad \exists \delta > 0 \quad \forall x, x_i \in D \quad \rho(x, x_i) < \delta \quad \rho(f(x), f(x_i)) < \varepsilon.
	\]
	А из этого напрямую следует, что если мы возьмём \(\delta\)-сеть для \(D\), то её образ будет являться \(\varepsilon\)-сетью для \(f(D)\).
\end{proof}

\begin{lemma}
	Если \(D\) сверхограниченно, то и его замыкание \(Cl(D)\) сверхограниченно.
\end{lemma}

\begin{proof}
	ДОДЕЛАТЬ
\end{proof}

\begin{lemma}
	Множество \(D\) сверхограниченно тогда и только тогда, когда любая последовательность из \(D\) содержит фундаментальную подпоследовательность.
\end{lemma}

\begin{proof}
	Докажем необходимость и достаточность:
	\begin{enumerate}
		\item[\(\Rightarrow\)] Вспомним определение фундаментальной подпоследовательности: \[
		\forall \varepsilon > 0 \quad \exists N \quad \forall n, m > N \quad \rho(y_{n}, y_{m}) < \varepsilon.
		\]
		Построим \(E = \{x_0, x_1,\ldots, x_n\}\) ---  \(\frac{\varepsilon}{2}\)-сеть для \(D\). Рассмотрим последовательность \((y_n)\). Так как наша сеть конечная, понятно, что в каком-то шаре \(B_\varepsilon (x_i)\) содержится бесконечно много членов последовательности. Но тогда расстояние между ними всеми меньше \(\varepsilon\), а значит они и составляют искомую подпоследовательность.
		\item[\(\Leftarrow\)] Пусть для какого-то \(\varepsilon\) не существует конечной \(\varepsilon\)-сети. Тогда есть такая последовательность, что в каждом шаре содержится только конечное число её членов. А из этого напрямую следует отрицание определения фундаментальной последовательности: \[
		\forall N \quad \exists n, m > N \quad \rho(y_{n}, y_{m}) \geqslant \varepsilon.
		\]
	\end{enumerate}
\end{proof}

\subsection{Компактность и конечные эпсилон-сети}

\begin{theorem}
	Пусть \(D \subset X\), где \(X\) --- полное метрическое пространство. Тогда следующие утверждения равносильны:
	\begin{enumerate}
		\item \(D\) компактно,
		\item \(D\) сверхограниченно и замкнуто.
	\end{enumerate}
\end{theorem}

\begin{proof}
	Докажем необходимость и достаточность:
	\begin{enumerate}
		\item[\(\Rightarrow\)] В метрическом пространстве компактность равносильна секвенциальной компактности. То есть из любой последовательности мы можем извлечь сходящуюся, а значит, и фундаментальную, подпоследовательность, что равносильно сверхограниченности. Замкнутость следует из компактности.
		\item[\(\Leftarrow\)] Так как \(D\) сверхограниченно, любая последовательность содержит фундаментальную подпоследовательность, которая в силу полноты \(X\) сходится к элементу из \(X\). То есть из любой последовательности мы можем извлечь сходяющуся подпоследовательность, что означает секвенциальную компактность \(D\), которая в метрическом пространстве равносильна компактности.
	\end{enumerate}
\end{proof}

\subsection{Простейшие свойства несобственного интеграла}

\hypertarget{svva}{}
\begin{theorem}
	Пусть функция \(f \colon [a, b) \to \mathbb{R}\) допустима. Тогда:
	\begin{enumerate}
		\item Для любого \(c \in (a, b)\) сходимость \(\int_a^{\to b} f\) равносильна сходимости \(\int_c^{\to b} f\) и, если интегралы сходятся, \[
		\int_a^{\to b} f = \int_a^c f + \int_c^{\to b} f.
		\]
		\item Если \(g\) допустима, интегралы \(\int_a^{\to b} f\) и \(\int_a^{\to b} g\) сходятся, то для любого \(\lambda \in \mathbb{R}\) интегралы \(\int_a^{\to b} \lambda f\) и \(\int_a^{\to b} f \pm g\) также сходятся, причём:
		\begin{enumerate}
			\item \(\displaystyle\int_a^{\to b} \lambda f = \lambda \displaystyle\int_a^{\to b} f\),
			\item \(\displaystyle\int_a^{\to b} f \pm g = \displaystyle\int_a^{\to b} f \pm \displaystyle\int_a^{\to b} g\).
		\end{enumerate}
		\item Если \(g\) допустима, интегралы \(\int_a^{\to b} f\) и \(\int_a^{\to b} g\) существуют в \(\overline{\mathbb{R}}\) (необязательно сходятся) и \(f \leqslant g\) на \([a, b)\), то \[
		\int_a^{\to b} f \leqslant \int_a^{\to b} g.
		\]
		\item Если \(f, g\) дифференцируемы на \([a, b)\) и \(f', g'\) допустимы, то \[
		\int_a^{\to b} f'g = fg \bigg|_a^{\to b} - \int_a^{\to b} fg'.
		\]
		Это надо понимать так: если существуют два конечных предела из трёх, то третий предел также существует и конечен, причём имеет место вышеизложенное равенство.
		\item Если \(\varphi \colon [\alpha, \beta) \to \langle A, B \rangle\), \(\varphi'\) допустима, то \[
		\int_a^{\to b} f(\varphi) \varphi' = \int_{\varphi(\alpha)}^{\varphi(\beta-)} f.
		\]
		Это надо понимать так: если существует один из интегралов, то существует и другой и имеет место вышеизложенное равенство.
	\end{enumerate}
\end{theorem}

\begin{proof}
	Все свойства доказываются почти одинаковым образом.
	\begin{enumerate}
		\item Для любого \(A \in (c, b)\) по свойству аддитивности определённого интеграла \[
		\int_a^A f = \int_a^c f + \int_c^A f.
		\]
		При \(A \to b-\) предел обеих частей равенства существует или нет одновременно. Перейдём к пределу и получим, что требуется.
		\item Для любого \(A \in [a, b)\) по свойству линейности определённого интеграла \[
		\int_a^A f \pm g = \int_a^A f \pm \int_a^A g \qquad \text{и} \qquad \int_a^A \lambda f = \lambda \int_a^A f.
		\]
		Тогда соответствующие интегралы сходятся, а после предельного перехода получаются необходимые равенства.
		\item Для любого \(A \in [a, b)\) из интегрирования неравенств \[
		\int_a^A f \leqslant\int_a^A g.
		\]
		Перейдём к пределу и получим, что нужно.
		\item Для любого \(A \in [a, b)\) справедливо \[
		\int_a^A f'g = fg \bigg|_a^A - \int_a^A fg'.
		\]
		Сделаем предельный переход и посмотрим, что вышло.
		\item Не будем доказывать, потому что тогда придётся всё аккуратно расписывать, а мы не хотим.
	\end{enumerate}
\end{proof}

\begin{remark}
	Так как мы только что убедились, что свойства несобственного интеграла один в один похожи на свойства определённого, \textit{стрелочку} писать больше \textit{не будем}.
\end{remark}

\subsection{\itshape Признаки сравнения сходимости несобственного интеграла}

\hypertarget{srshlem}{}
\begin{lemma}
	Пусть функция \(f\) допустима на \([a, b)\), \(f \geqslant 0\). Обозначим \linebreak \(\Phi(A) = \displaystyle\int_a^A f(x) \, dx\), где \(A \in [a, b)\). Тогда следующие утверждения равносильны:
	\begin{enumerate}
		\item \(\displaystyle\int_a^b f(x) \, dx\) сходится,
		\item \(\Phi(A)\) ограничена.
	\end{enumerate}
\end{lemma}

\begin{proof}
	Нужно проверить, что конечный предел \(\lim\limits_{A \to b-} \Phi(A)\) существует тогда и только тогда, когда \(\Phi(A)\) ограничена. Понятно, что если предел существует, то функция ограничена. Также очевидно, что \(\Phi(A)\) возрастает. Значит, по теореме о пределе монотонной функции, её предел существует и конечен. 
\end{proof}

\hypertarget{priz}{}
\begin{theorem}
	Пусть функции \(f, g\) допустимы на \([a, b)\), \(f, g \geqslant 0\). Тогда справедливы утверждения:
	\begin{enumerate}
		\item Пусть \(f \leqslant g\) на \([a, b)\). Тогда:
		\begin{enumerate}
				\item \(\int_a^b g\) сходится \(\Rightarrow\) \(\int_a^b f\) сходится,
			\item \(\int_a^b f\) расходится \(\Rightarrow\) \(\int_a^b g\) расходится.
		\end{enumerate}
		\item Пусть \(\lim\limits_{x \to b-} \dfrac{f(x)}{g(x)} = l\). Тогда:
		\begin{enumerate}
			\item Если \(l = +\infty\), то:
			\begin{enumerate}
				\item \(\int_a^b f\) сходится \(\Rightarrow\) \(\int_a^b g\) сходится,
				\item \(\int_a^b g\) расходится \(\Rightarrow\) \(\int_a^b f\) расходится.
			\end{enumerate}
			\item Если \(l = 0\), то:
			\begin{enumerate}
				\item \(\int_a^b g\) сходится \(\Rightarrow\) \(\int_a^b f\) сходится,
				\item \(\int_a^b f\) расходится \(\Rightarrow\) \(\int_a^b g\) расходится.
			\end{enumerate}
			\item Если \(l \in (0, +\infty)\), то \(\int_a^b f\) и \(\int_a^b g\) сходятся и расходятся одновременно.
		\end{enumerate}
	\end{enumerate}
\end{theorem}

\begin{proof}
	Докажем утверждения, которые мы наплодили.
	\begin{enumerate}
		\item Очевидно следует из \hyperlink{srshlem}{леммы}.
		\item
		\begin{enumerate}
			\item Если \(\lim\limits_{x \to b-} \dfrac{f(x)}{g(x)} = +\infty\), то, НСНМ, \(g \leqslant f\). Отсылаем к пункту~1.
			\item Если \(\lim\limits_{x \to b-} \dfrac{f(x)}{g(x)} = 0\), то, НСНМ, \(f \leqslant g\). Опять отсылаем к пункту~1.
			\item Существует такое \(c \in [a, b)\), что для всех \(x \in [c, b)\) выполняется, 	например, \[
				\frac{1}{2} \, l < \frac{f(x)}{g(x)} < \frac{3}{2} \, l,
			\]
			то есть \[
				\frac{1}{2} \, l  \cdot g(x) < f(x) < \frac{3}{2} \, l \cdot g(x).
			\]
			Учитывая, что сходимость на \([a, b)\) равносильна сходимости на \([c, b)\), по пункту~1 из правой части получаем, что если \(\int_a^b g\) сходится, то и \(\int_a^b f\) сходится, а из левой --- что если \(\int_a^b f\) сходится, то и \(\int_a^b g\) сходится. Получили равносильность.
		\end{enumerate}
	\end{enumerate}
\end{proof}

\subsection{Изучение сходимости интеграла $\int_{10}^\infty \frac{dx}{x^\alpha (\ln x)^\beta}$}

\begin{example}
	Исследуем интеграл $\displaystyle \int_{10}^{+\infty} \frac{dx}{x^\alpha (\ln x)^\beta}$ на сходимость.
	
	Рассотрим, как на сходимость влияет \(\alpha\):
	\begin{enumerate}
		\item Пусть \(\alpha > 1\). Запишем \(\alpha = 1 + 2a\), где \(a > 0\). Тогда \[
			\frac{1}{x^\alpha (\ln x)^\beta} = \frac{1}{x^{1+a}} \cdot \frac{1}{x^a (\ln x)^\beta}.
		\]
		Ясно, что \(\dfrac{1}{x^a (\ln x)^\beta} \to 0\) при \(x \to \infty\). Почему? При положительном \(a\) выражение \(x^a \to \infty\) при \(x \to \infty\), а логарифм не сможет этому помешать: при \(\beta \geqslant 0\) он тоже стремится к бесконечности, а при \(\beta < 0\) логарифм просто недостаточно быстро растёт --- проверить это можно, вычислив по \hyperlink{t3}{правилу Лопиталя} \(\lim\limits_{x \to \infty} \dfrac{(\ln x)^{-\beta}}{x^a} = \lim\limits_{x \to \infty} \left(\dfrac{\ln x}{x^{-\frac{a}{\beta}}} \right)^{-\beta}\).
		Тогда, НСНМ, \[
			\frac{1}{x^{1 + a} (\ln x)^\beta} \cdot \frac{1}{x^a (\ln x)^\beta} \leqslant 	\frac{1}{x^{1 + a}}.
		\]
		Интеграл \(\displaystyle \int_{10}^{+\infty} \frac{1}{x^{1 + a}}\) сходится как эталонный, а значит и рассматриваемый интеграл тоже сходится.
		\item Пусть теперь \(\alpha < 1\). Запишем \(\alpha = 1 - 2b\), где \(b > 0\). Тогда \[
			\frac{1}{x^\alpha (\ln x)^\beta} = \frac{1}{x^{1 - b}} \cdot \frac{x^b}{(\ln x)^\beta}.
		\]
		Интеграл от \(\dfrac{1}{x^{1 - b}}\) расходится, а \(\dfrac{x^b}{(\ln x)^\beta}\) этому не мешает, так как стремится к бесконечности при \(x \to \infty\) (опять потому что логарифм растёт медленнее степенной функции).
		Тогда, НСНМ, \[
			\frac{1}{x^\alpha (\ln x)^\beta} \geqslant \frac{1}{x^{1 - b}}.
		\]
		Интеграл \(\displaystyle \int_{10}^{+\infty} \frac{1}{x^{1 - b}}\) расходится как эталонный, а значит и рассматриваемый интеграл тоже расходится.
		\item Пусть \(\alpha = 1\), то есть мы исследуем интеграл \[
			\int_{10}^\infty \frac{dx}{x (\ln x)^\beta}
		\]
		Сделаем замену \(y = \ln x\): \[
			\int_{\ln 10}^{+\infty} \frac{d(e^y)}{e^y y^\beta} = \int_{\ln 10}^{+\infty} \frac{d(y)}{y^\beta}.
		\]
		Это эталонный интеграл, а значит он сходится при  \(\beta > 1\) и расходится при  \(\beta \leqslant 1\).
	\end{enumerate}
\end{example}

\subsection{Интеграл Эйлера -- Пуассона}

\hypertarget{puas}{}
\begin{theorem}
	\[
		\int_0^{+\infty} e^{-x^2} \, dx = \frac{\sqrt{\pi}}{2}.
	\]
\end{theorem}

\begin{proof}
	Из выпуклости функции \(e^x\) следует, что \(e^x \geqslant 1 + x\) (просто провели касательную в точке \(0\)), откуда, в свою очередь, следует неравенство \[
		1 - x^2 \leqslant e^{-x^2} \leqslant \frac{1}{1 + x^2}
	\]
	(левая часть очевидна, правая следует из того, что \(e^{x^2} \geqslant 1 + x^2\)).
	
	Возведём левую часть в \(n\)-ю степень и проинтегрируем на отрезке \([0, 1]\): \[
		\int_0^1 (1 - x^2)^n \, dx \leqslant \int_0^1 e^{-nx^2} \, dx,
	\]
	а правую тоже возведём в степень \(n\) и проинтегрируем уже на \([0, +\infty]\): \[
			\int_0^{+\infty} e^{-nx^2} \, dx \leqslant \int_0^{+\infty} \left(\frac{1}{1 + x^2} \right)^n \, dx.
	\]
	Так как функция \(e^x\) положительна, можем объединить эти два неравенства следующим образом: \[
		\int_0^1 (1 - x^2)^n \, dx \leqslant \int_0^1 e^{-nx^2} \, dx \leqslant \int_0^{+\infty} e^{-nx^2} \, dx \leqslant \int_0^{+\infty} \left(\frac{1}{1 + x^2} \right)^n \, dx.
	\]
	Левый центальный интеграл опустим, он не пригодится: \[
		\int_0^1 (1 - x^2)^n \, dx \leqslant \int_0^{+\infty} e^{-nx^2} \, dx \leqslant \int_0^{+\infty} \left(\frac{1}{1 + x^2} \right)^n \, dx.
	\]
	Теперь посчитаем получившиеся интегралы:
	\begin{enumerate}
		\item В левом сделаем замену \(x = \cos t\): \[
			\int_0^1 (1 - x^2)^n \, dx = \int_0^{\frac{\pi}{2}} (\sin t)^{2n + 1} \, dt;
		\]
		по \hyperlink{vallem}{лемме из формуле Валлиса} \[
			\int_0^{\frac{\pi}{2}} (\sin t)^{2n + 1} \, dt = \frac{(2n)!!}{(2n + 1)!!}.
		\]
		\item В правом сделаем замену \(x = \tg t\): \[
			\int_0^{+\infty} \left(\frac{1}{1 + x^2} \right)^n \, dx = \int_{0}^{\frac{\pi}{2}} (\cos t)^{2n - 2} \, dt;
		\]
		опять по \hyperlink{vallem}{лемме из формуле Валлиса} \[
			\int_{0}^{\frac{\pi}{2}} (\cos t)^{2n - 2} \, dt  = \frac{(2n - 3)!!}{(2n - 2)!!} \cdot \frac{\pi}{2}.
		\]
		\item В центральном сделаем замену \(x = \dfrac{t}{\sqrt{n}}\): \[
			\int_0^{+\infty} e^{-nx^2} \, dx = \frac{1}{\sqrt{n}} \int_0^{+\infty} e^{-t^2} \, dt.
		\]
	\end{enumerate}
	Имеем \[
		\frac{(2n)!!}{(2n + 1)!!} \leqslant \frac{1}{\sqrt{n}} \int_0^{+\infty} e^{-t^2} \, dt \leqslant \frac{(2n - 3)!!}{(2n - 2)!!} \cdot \frac{\pi}{2}.
	\]
	Домножим неравенство на \(\sqrt{n}\): \[
		\frac{(2n)!!}{(2n + 1)!!} \cdot \sqrt{n} \leqslant \int_0^{+\infty} e^{-t^2} \, dt \leqslant \frac{(2n - 3)!!}{(2n - 2)!!} \cdot \frac{\pi}{2} \cdot \sqrt{n}.
	\]
	Иначе говоря, \[
		\frac{n}{2n + 1} \cdot \frac{(2n)!!}{(2n - 1)!!} \cdot \frac{1}{\sqrt{n}} \leqslant \int_0^{+\infty} e^{-t^2} \, dt \leqslant \frac{1}{\cfrac{(2n - 2)!!}{(2n - 3)!!} \cdot \cfrac{1}{\sqrt{n - 1}}} \cdot \frac{\sqrt{n}}{\sqrt{n - 1}}\cdot \frac{\pi}{2}.
	\]
	Так как неравентсво справедливо для любого \(n\), устремим его в бесконечность. По \hypertarget{wall}{формуле Валлиса} получим, что выражения слева и справа стремятся к \(\dfrac{\sqrt{\pi}}{2}\) при \(n \to \infty\). А значит, наш интеграл равен \(\dfrac{\sqrt{\pi}}{2}\).
\end{proof}

\subsection{Опорная прямая}

\begin{definition}
	Пусть \(f \colon \langle a, b \rangle \to \mathbb{R}, \ x_0 \in \langle a, b \rangle\). Прямая, задаваемая уравнением \(y = \ell(x)\), называется \textit{опорной} для функции  \(f\) в точке \(x_0\), если \[
		f(x_0) = \ell(x_0) \quad \text{и} \quad f(x) \geqslant \ell(x) \quad \text{для всех} \ x \in \langle a, b \rangle. 
	\]
	Если же \[
		f(x_0) = \ell(x_0) \quad \text{и} \quad f(x) > \ell(x) \quad \text{для всех} \ x \in \langle a, b \rangle \setminus \{x_0\},
	\]
	то прямая называется \textit{строго опорной} для функции \(f\) в точке \(x_0\).
	
	Другими словами, прямая называется опорной к \(f\) в точке \(x_0\), если она проходит через точку \((x_0, f(x_0))\) и лежит не выше графика функции. Строго опорная прямая лежит ниже графика функции во всех точках, кроме \((x_0, f(x_0))\).
\end{definition}

\subsection{Кусочно-непрерывная функция, интеграл от неё}

\begin{definition}
	Функция \(f \colon [a, b] \to \mathbb{R}\) называется \textit{кусочно-непрерывной}, если она непрерывна всюду, кроме конечного числа точек, в которых имеет разрывы первого рода.
\end{definition}

\begin{definition}
	Пусть \(f\) --- кусочно-непрерывная функция, \(\widetilde{f}_k\) --- её сужение на промежуток \([x_{k - 1}, x_k]\), то есть \[
		\widetilde{f}_k = f \big|_{[x_{k - 1}, x_k]} =
		\begin{cases}
			f(x_{k - 1} + 0), & x = x_{k - 1} 		   \\
			f(x),			  & x \in (x_{k - 1}, x_k) \\
			f(x_k - 0),		  & x = x_k.
		\end{cases}
	\]
	Тогда \[
		\int_a^b f = \sum_{k = 1}^n \int_{x_{k - 1}}^{x_k} \widetilde{f}_k
	\]
	называется \textit{интегралом кусочно-непрерывной функции} \(f\).
\end{definition}

\subsection{Почти первообразная}

\begin{definition}
	Пусть \(F, f \colon [a, b] \to \mathbb{R}\). Функция \(F\) называется \textit{почти первообразной} \(f\) на \([a, b]\), если \(F'(x) = f(x)\) для всех \(x \in [a, b]\), кроме конечного числа.
\end{definition}

\begin{remark}
	Если \(f\) --- кусочно-непрерывная функция, у неё существует почти первообразная.
\end{remark}

\begin{proof}
	Пусть \(F_k\) --- первообразная \(\widetilde{f}_k\) на \([x_{k - 1}, x_k]\).
	
	ДОДЕЛАТЬ
\end{proof}

\begin{remark}
	Пусть \(f \colon [a, b] \to \mathbb{R}\) --- кусочно-непрерывная функция, \(F\) --- почти первообразная \(f\) на \([a, b]\). Тогда \[
		\int\limits_a^b f = F(b) - F(a).
	\]
\end{remark}

\begin{proof}
	Пусть \(a = x_0 < x_1 < \ldots < x_{n - 1} < x_n = b\). Тогда \[
		\int_a^b = \sum_{k = 1}^n \int_{x_{k - 1}}^{x_k} \widetilde{f}_k = \sum_{k = 1}^n F_k(x_k) - F_k (x_{k - 1}) = \sum_{k = 1}^n F(x_k) - F(x_{k - 1}).
	\]
	Получили телескопическую сумму, после приведения подобных равную \(F(a) - F(b)\).
\end{proof}

\subsection{Функция промежутка, аддитивная функция промежутка}

\begin{definition}
	Пусть \(a, b \in \overline{\mathbb{R}}\), \(a < b\). Обозначим как \(Segm \langle a, b \rangle\) множество вида \(\{[p, q] \mid [p, q] \subset \langle a, b \rangle\}\). Тогда:
	\begin{enumerate}
		\item Функция \(\Phi \colon Segm \langle a, b \rangle \to \mathbb{R}\) называется \textit{функцией промежутка},
		\item Функция \(\Phi \colon Segm \langle a, b \rangle \to \mathbb{R}\), такая, что \[
			\forall [p, q] \in Segm \langle a, b \rangle \quad \forall \in (p, q) \quad \Phi([p, q]) = \Phi([p, c]) + \Phi([c, q]),
		\]
		называется \textit{аддитивной функцией промежутка}.
	\end{enumerate}
\end{definition}

\begin{example}
	Если функция \(f\) непрерывна на \(\langle a, b \rangle\), то \(f\) --- плотность функции \(\Phi\), такой, что \[
		\Phi([p, q]) = \int\limits_a^b f.
	\]
	Действительно, по \hyperlink{sredneye}{теореме о среднем} имеем\[
		\inf_{[p, q]} (f) \cdot (q - p) \leqslant \int\limits_p^q f \leqslant \max_{[p, q]} (f) \cdot (q - p).
	\]
\end{example}

\subsection{Плотность аддитивной функции промежутка}

\begin{definition}
	Пусть \(\Phi \colon Segm \langle a, b \rangle \to \mathbb{R}\) --- аддитивная функция промежутка, тогда функция \(f: \langle a, b \rangle \to \mathbb{R}\) называется \textit{плотностью} \(\Phi\) на \(\langle a, b \rangle\), если \[
	\forall \Delta \in Segm \langle a, b \rangle \quad |\Delta| \inf_\Delta(f) \leqslant \Phi(\Delta) \leqslant |\Delta| \sup_\Delta(f).
	\]
	
\end{definition}

\subsection{\itshape Верхний и нижний пределы последовательности}

\begin{definition}
	Пусть \((x_n)\) --- вещественная последовательность. Тогда последовательность \((y_n)\), где \(y_n = \sup (x_n, x_{n + 1}, x_{n + 2}, \ldots)\), называется \textit{верхней огибающей} \((x_n)\), а \((z_n)\), где \(z_n = \inf (x_n, x_{n + 1}, x_{n + 2}, \ldots)\), --- \textit{нижней огибающей} \((x_n)\).
\end{definition}

\begin{remark}
	Некоторые свойства огибающих:
	\begin{enumerate}
		\item Для всех \(n\) \(z_n \leqslant x_n \leqslant y_n\),
		\item Так как \(y_n \geqslant y_{n + 1} \geqslant y_{n + 2} \geqslant \ldots\) и \(z_n \leqslant z_{n + 1} \leqslant z_{n + 2} \leqslant \ldots\), то для всех~\(n\) \(z_1 \leqslant y_n\) и \(z_n \leqslant y_1\).
	\end{enumerate}
\end{remark}

\begin{definition}
	Пусть \((x_n)\) --- вещественная последовательность, \((y_n)\) --- её верхняя огибающая, а \((z_n)\) --- нижняя. Тогда \[
	\varlimsup_{n \to \infty} x_n = \lim_{n \to \infty} y_n \in \overline{\mathbb{R}}
	\]
	называется \textit{верхним пределом} \((x_n)\), а \[
	\varliminf\limits_{n \to \infty} x_n = \lim\limits_{n \to \infty} z_n \in \overline{\mathbb{R}} \ \textit{---}
	\]
	\textit{нижним пределом}.
\end{definition}

\begin{example}
	Если \(x_n = (-1)^n\), то \(y_n \equiv 1\), \(z_n \equiv -1\), то есть \(\varlimsup\limits_{n \to \infty} = 1\), \(\varliminf\limits_{n \to \infty} = -1\).
\end{example}

\subsection{Частичный предел}

\begin{definition}
	Пусть \((x_n)\) --- последовательность в (метрическом пространстве) \(X\). Тогда \(l \in X\) называется \textit{частичным пределом} \((x_n)\), если существует строго возрастающая последовательность натуральных чисел \((n_k)\), такая, что \(x_{n_k} \to l\) при \(k \to \infty\).
\end{definition}

\begin{example}
	Если \(x_n = (-1)^n\), то \(1\) и \(-1\) --- частичные пределы \((x_n)\).
\end{example}

\subsection{Дробление отрезка, ранг дробления, оснащение}

\begin{definition}
	\textit{Дроблением} отрезка \([a, b]\) называется возрастающий набор \(x_i\), принадлежащих отрезку \([a, b]\). Пусть для определённости \linebreak \(a = x_0 \leqslant x_1 \leqslant \ldots \leqslant x_{n - 1} \leqslant x_n = b\). Тогда эти \(x_i\) разбивают \([a, b]\) на \(n\) соприкасающихся подотрезков \([x_i, x_{i + 1}]\).
\end{definition}

\begin{definition}
	\textit{Рангом} дробления называется длина наибольшего подотрезка \([x_i, x_{i + 1}]\), то есть \(\max (x_{i + 1} - x_i)\), где \(i \in 1 : n\).
\end{definition}

\begin{definition}
	\textit{Оснащением} называется набор точек \(\xi_i\), принадлежащих отрезку \([a, b]\), где \(i \in 1 : n\) и каждая \(\xi_i\) принадлежит уникальному подотрезку дробления.
\end{definition}

\subsection{Кривая Пеано}

\begin{example}
	ДОДЕЛАТЬ
\end{example}

\subsection{Гладкий путь, вектор скорости, носитель пути}

\begin{definition}
	Путь\footnote{Напомним, что \textit{путём} из \(A\) в \(B\), где \(A\) и \(B\) \(\in \mathbb{R}^m\), называется непрерывная функция \(\gamma \colon [a, b] \to \mathbb{R}^m\), такая, что \(\gamma(a) = A\), \(\gamma(b) = B\).} \(\gamma \colon [a, b] \to \mathbb{R}^m\) называется \textit{гладким}, если каждая из координатных функций \(\gamma_i\), где \(i \in 1 : m\), непрерывно дифференцируема.
\end{definition}

\begin{definition}
	\textit{Вектором скорости} называется производная пути.
\end{definition}

\begin{definition}
	\textit{Носителем пути} \(C_\gamma\) называется образ пути \(\gamma\).
\end{definition}
\subsection{Лемма о представлении синуса в виде конечного произведения}

\begin{lemma} \hypertarget{sinlem}{}
	Для всех \(x \in \mathbb{R}\) и для всех \(n \in \mathbb{N}\) \[
		\sin x = (2n + 1) \sin \frac{x}{2n + 1} \cdot \prod_{k=1}^{n} \left(1 - \frac{\sin^2 \frac{x}{2n + 1}}{\sin^2 \frac{\pi k}{2n + 1}} \right).
	\]
\end{lemma}
\begin{proof}
	Обозначим \(m = 2n + 1\) и выскочим в комплексную плоскость. По формулам Эйлера и Муавра имеем \[
		e^{i m z} = (\cos z + i \sin z)^m = \cos mz + i \sin mz.
	\]
	С другой стороны, \[
		(\cos z + i \sin z)^m = \sum_{k=1}^{m} C_m^k \cdot (\cos z)^{m-k} \cdot (i \sin z)^k.
	\]
	Сравним мнимые части: \[
		\sin mz = m \cdot (\cos z)^{m-1} \cdot \sin z - C_m^3 \cdot (\cos z)^{m-3} \cdot (\sin z)^3 + \ldots
	\]
	Так как \(m\) --- нечётное число, косинусы везде стоят в чётных степенях. Заменим везде \(\cos^2 x\) на \(1 - \sin^2 x\) и получим
	\begin{equation} \label{sinkon}
		\sin mz = \sin z \cdot P(\sin^2 z),
	\end{equation}
	где \(P\) --- многочлен степени \(n\), зависящий от \(\sin^2 x\).
	
	Теперь выберем \(z = \frac{\pi}{m}\), потом \(z = \frac{2 \pi}{m}\) и так до \(z = \frac{n \pi}{m}\) \big(заметим, что все эти числа лежат в \(\left(0, \frac{\pi}{2} \right)\)\big). Во всех этих случаях \(\sin mz = 0\), а \(\sin z \neq 0\). Значит, для всех выбранных \(z\) \(\sin^2 z\) --- корень многочлена \(P\). Так как многочлен имеет степень \(n\), мы можем его разложить: \[
		P(u) = C \cdot \left(u - \sin^2 \frac{\pi}{m} \right) \cdot \left(u - \sin^2 \frac{2 \pi}{m} \right) \cdot \ldots \cdot \left(u - \sin^2 \frac{n \pi}{m} \right),
	\]
	где \(C\) --- какая-то константа. Или мы можем записать это так: \[
		P(u) = C \cdot \left(1 - \frac{u}{\sin^2 \frac{\pi}{m}} \right) \cdot \left(1 - \frac{u}{\sin^2 \frac{2 \pi}{m}} \right) \cdot \ldots \cdot \left(1 - \frac{u}{\sin^2 \frac{n \pi}{m}} \right).
	\]
	Теперь, если мы подставим вместо \(u\) ноль, поймём, что \(C = P(0)\), что, в свою очередь, равно \(\lim\limits_{z \to 0} \frac{\sin mz}{\sin z} = m\) (заменим синус на эквивалентную). Обозначим \(x = mz = (2n + 1) z\) и из~\eqref{sinkon} сразу получим требуемую формулу.
\end{proof}

\subsection{Разложение синуса в бесконечное произведение}

\begin{theorem}
	Для любого \(x \in \mathbb{R}\) \[
		\sin x = x \cdot \prod_{k=1}^{\infty} \left(1 - \frac{x^2}{\pi^2 k^2} \right).
	\]
\end{theorem}
\begin{proof}
	Очевидно, что для \(x = \pi l\) при \(l \in \mathbb{Z}\) утверждение выполняется. Рассмотрим все прочие \(x\). Вспомним \hyperlink{sinlem}{лемму}:
	\begin{equation} \label{sinbes}
		\sin x = (2n + 1) \sin \frac{x}{2n + 1} \cdot \prod_{k=1}^{n} \left(1 - \frac{\sin^2 \frac{x}{2n + 1}}{\sin^2 \frac{\pi k}{2n + 1}} \right).
	\end{equation}
	Устремим \(n\) к бесконечности и рассмотрим \(k\)-ю скобку: \[
		1 - \frac{\sin^2 \frac{x}{2n + 1}}{\sin^2 \frac{\pi k}{2n + 1}} \to 1 - \frac{x^2}{\pi^2 k^2}.
	\]
	Тогда~\eqref{sinbes} можно записать как \[
		\sin x = (2n + 1) \sin \frac{x}{2n + 1} \cdot \prod_{l=1}^{k} \left(1 - \frac{\sin^2 \frac{x}{2n + 1}}{\sin^2 \frac{\pi l}{2n + 1}} \right) \cdot \prod_{l=k+1}^{n} \left(1 - \frac{\sin^2 \frac{x}{2n + 1}}{\sin^2 \frac{\pi l}{2n + 1}} \right).
	\]
	Введём обозначения:
	\begin{align*}
		u_k^n &= (2n + 1) \sin \frac{x}{2n + 1} \cdot \prod_{l=1}^{k} \left(1 - \frac{\sin^2 \frac{x}{2n + 1}}{\sin^2 \frac{\pi l}{2n + 1}} \right), \\
		V_k^n &= \prod_{l=k+1}^{n} \left(1 - \frac{\sin^2 \frac{x}{2n + 1}}{\sin^2 \frac{\pi l}{2n + 1}} \right).
	\end{align*}
	Получаем, что \(\displaystyle u_k^n \xrightarrow[n \to \infty]{} x \cdot \prod_{l=1}^{k} \left(1 - \frac{x^2}{\pi^2 l^2} \right)\). Обозначим этот предел \(u_k\). Значит, существует конечный предел \(\lim\limits_{n \to \infty} V_k^n\), обозначим его \(V_k\). Получаем, что \(\sin x = u_k \cdot V_k\).
	
	Пусть теперь \(k \to \infty\). Очевидно, \(\displaystyle \lim_{k \to \infty} u_k = x \cdot \prod_{l=1}^{\infty} \left(1 - \frac{x^2}{\pi^2 l^2} \right)\). Осталось проверить, что \(V_k \xrightarrow[k \to \infty]{} 1\).
	
	Заметим, что \(\dfrac{2}{\pi} \cdot \varphi \leqslant \sin \varphi \leqslant \varphi\) (ДОДЕЛАТЬ) при \(\varphi \in \left[0, \frac{\pi}{2} \right]\). Тогда ясно, что \[
		1 \geqslant 1 - \frac{\sin^2 \frac{x}{2n + 1}}{\sin^2 \frac{\pi l}{2n + 1}} \geqslant 1 - \frac{\frac{x^2}{(2n + 1)^2}}{\frac{4 \pi^2 l^2}{\pi^2 (2n + 1)^2}} = 1 - \frac{x^2}{4 l^2},
	\]
	а следовательно, \[
		1 \geqslant V_k^n \geqslant \prod_{l=k+1}^{n} \left(1 - \frac{x^2}{4 l^2} \right)
	\]
	(здесь \(n\) и \(k\) должны быть достаточно большими, чтобы условное \(\varphi\) лежало в \(\left(0, \frac{\pi}{2} \right)\) ДОДЕЛАТЬ).
	Теперь устремим \(n\) в бесконечность и получим \[
		1 \geqslant V_k \geqslant \prod_{l=k+1}^{+\infty} \left(1 - \frac{x^2}{4 l^2} \right) \to 1.
	\]
\end{proof}

\subsection{Теорема о двойных и повторных пределах}

ДОДЕЛАТЬ

\subsection{Единственность производной}


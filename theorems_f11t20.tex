\subsection{Неравенство Чебышёва}

% неравенство Чебышёва
\begin{theorem}
	Пусть \(f, g\) непрерывны на \([a, b]\) и монотонны, причём одинаково монотонны. Пусть \(I_f\) --- \hyperlink{average}{среднее значение} функции \(f\). Тогда \(I_f \cdot I_g \le I_{fg}\).
\end{theorem}

\begin{proof}
	Так как \(f, g\) монотонны одинаково, справедливо следующее: \[
		\forall x, y \in [a, b] \quad (f(x) - f(y)) \cdot (g(x) - g(y)) \ge 0.
	\]
	Раскроем скобки: \[
		f(x)g(x) - f(x)g(y) - f(y)g(x) + f(y)g(y) \ge 0.
	\]
	Проинтегрируем неравенство по \(x\) на \([a, b]\): \[
		\int_a^b fg - \int_a^b f \cdot g(y) - f(y) \cdot \int_a^b g + (b - a) \cdot f(y)g(y) \ge 0
	\]
	и разделим на \(b - a\): \[
		I_{fg} - I_f \cdot g(y) - f(y) \cdot I_g + f(y)g(y) \ge 0.
	\]
	Теперь проинтегрируем по \(y\) на том же промежутке и опять разделим на \(b - a\): \[
		I_{fg} - I_f \cdot I_g - I_f \cdot I_g + I_{fg} \ge 0.
	\]
	Приведём подобные и получим требуемый результат.
\end{proof}

\subsection{Иррациональность числа пи}

% иррациональность числа пи

\begin{theorem}
	Число пи иррационально.
\end{theorem}

\begin{proof}
	Рассмотрим последовательность \((H_n)\): \[
		H_n = \frac{1}{n!} \int\limits_{-\frac\pi2}^{\frac\pi2}
		\left(\frac{\pi^2}4 - t^2 \right)^n \cos(t) dt.
	\]
	Нам интересно выразить \(n\)-й член последовательности через предыдущие. Сперва проинтегрируем \(H_n\) по частям: представим подынтегральное выражение как \(fg'\),
	где \(f = \left(\frac{\pi^2}4 - t^2 \right)^n\), а \(g' = \cos(t)\). Следовательно, \(f' = -2t \cdot n \cdot \left(\frac{\pi^2}4 - t^2 \right)^{n-1}\), а \(g = \sin(t)\). Тогда \[
		H_n = \frac1{n!} \left(\frac{\pi^2}4 - t^2 \right)^n \sin(t)
		\bigg|_{-\frac\pi2}^{\frac\pi2}  - \frac2{(n - 1)!} \int_{-\frac\pi2}^{\frac\pi2}
		t \left(\frac{\pi^2}4 - t^2 \right)^{n-1} \sin(t) dt.
	\]
	Первое слагаемое зануляется, оставшееся опять проинтегрируем по частям: на этот раз
	\(f = t \left(\frac{\pi^2}4 - t^2 \right)^{n-1}\), а \(g' = \sin(t)\). Следовательно, \\
	\(f' = \left(\frac{\pi^2}4 - t^2 \right)^{n-1} - 2t^2 (n - 1) \left(\frac{\pi^2}4 - t^2 \right)^{n-2}\), а \(g = -\cos(t)\). Немного подправим \(f'\) --- во втором слагаемом вместо множителя \(t^2\)
	запишем \\
	\(\left(t^2 - \frac{\pi^2}4 \right) + \frac{\pi^2}4\), то есть \(f'\)
	примет такой вид:
	\begin{align*}
		\left(\frac{\pi^2}4 - t^2 \right)^{n - 1}
		&- 2\left(\left(t^2 - \frac{\pi^2}4 \right)
		+ \frac{\pi^2}4 \right) (n - 1) \left(\frac{\pi^2}4 - t^2 \right)^{n - 2}, \\
		\hbox{или } \left(\frac{\pi^2}4 - t^2 \right)^{n-1}
		&+ 2\left(\left(\frac{\pi^2}4 - t^2 \right)
		- \frac{\pi^2}4 \right) (n - 1) \left(\frac{\pi^2}4 - t^2 \right)^{n - 2}.
	\end{align*}
	Раскроем скобки: \[
		\left(\frac{\pi^2}4 - t^2 \right)^{n - 1}
		+ 2 (n - 1) \left(\frac{\pi^2}4 - t^2 \right)^{n - 1}
		- 2 (n - 1) \frac{\pi^2}4 \left(\frac{\pi^2}4 - t^2 \right)^{n - 2},
	\]
	и приведём подобные: \[
		(2n - 1) \left(\frac{\pi^2}4 - t^2 \right)^{n - 1}
		- (n - 1) \frac{\pi^2}2 \left(\frac{\pi^2}4 - t^2 \right)^{n - 2}.
	\]
	Запишем, наконец, результат интегрирования по частям
	\bigg(\(fg \bigg|_{-\frac\pi2}^{\frac\pi2}\) снова занулится, поэтому его писать не будем \bigg):
	\begin{align*}
		H_n &= -\frac2{(n - 1)!} \int_{-\frac\pi2}^{\frac\pi2}
		(2n - 1) \left(\frac{\pi^2}4 - t^2 \right)^{n - 1} (-\cos(t)) dt + \ldots \\
		\ldots &+ \frac2{(n - 1)!} \int_{-\frac\pi2}^{\frac\pi2}
		(n - 1) \frac{\pi^2}2 \left(\frac{\pi^2}4 - t^2 \right)^{n - 2} (-\cos(t)) dt = \ldots \\
		\ldots &= \frac{4n - 2}{(n - 1)!} \int_{-\frac\pi2}^{\frac\pi2}
		\left(\frac{\pi^2}4 - t^2 \right)^{n - 1} \cos(t) dt - \ldots \\
		\ldots &- \frac{\pi^2}{(n - 2)!} \int_{-\frac\pi2}^{\frac\pi2} \left(\frac{\pi^2}4 - t^2 \right)^{n - 2} \cos(t) dt = \ldots \\
		\ldots &= (4n - 2) H_{n - 1} + \pi^2 H_{n - 2}.
	\end{align*}
	Итак, мы нашли рекуррентную формулу для \(H_n\). Теперь мы можем выразить любой член последовательности, кроме нулевого и первого. Вычислим их непосредственно
	(при вычислении \(H_1\) придётся два раза проинтегрировать по частям):
	\begin{align*}
		H_0 &= \int_{-\frac\pi2}^{\frac\pi2} \cos(t) dt = 2, \\
		H_1 &= \int_{-\frac\pi2}^{\frac\pi2}
		\left(\frac{\pi^2}4 - t^2 \right) \cos(t) dt =
		\left(\frac{\pi^2}4 - t^2 \right) \sin(t)
		\bigg|_{-\frac\pi2}^{\frac\pi2} + \int_{-\frac\pi2}^{\frac\pi2} 2t \sin(t) dt = \ldots \\
		\ldots &= 2 \int_{-\frac\pi2}^{\frac\pi2} t \sin(t) dt = -2t \cos(t) \bigg|_{-\frac\pi2}^{\frac\pi2} + 2 \int_{-\frac\pi2}^{\frac\pi2} cos(t) dt = 4.
	\end{align*}
	Используем же, наконец, всё это, чтобы доказать, что \(\pi\) (и даже \(\pi^2\)) иррационально. Заметим, что \(H_n\) --- многочлен с целыми коэффициентами от \(\pi^2\), причём степени не больше, чем \(n\). Действительно, чтобы по полученной ранее рекурентной формуле разложить \(H_n\) до целых чисел \(H_0\) и \(H_1\), потребуется применить её \(n - 1\) раз, соответственно \(\pi^2\) может входить в получившийся в результате разложения многочлен в степени уж точно не больше, чем \(n\). Обозначим этот многочлен как \(P_n(\pi^2)\).
	
	Предположим теперь, что \(\pi^2\) --- рациональное число, то есть \hbox{\(\pi^2 = \frac{p}{q}\)}, где \(p, q \in \mathbb{N}\). Заметим, что тогда \(q^n P_n(\pi^2)\) --- целое число (так как домножением на \(q^n\) мы сократили все знаменатели). Мы также знаем, что \(P_n(\pi^2) = H_n > 0\), так как подынтегральная функция в \(H_n\) равна нулю в точках \(-\frac\pi2\) и \(\frac\pi2\) и положительна в остальных. Следовательно, \(q^n P_n(\pi^2)\) --- это как минимум единица. Запишем подробнее: \[
		q^n P_n(\pi^2) = q^n H_n = \frac{q^n}{n!} \int\limits_{-\frac\pi2}^{\frac\pi2}
		\left(\frac{\pi^2}4 - t^2 \right)^n \cos(t) dt \ge 1.
	\] 
	Оценим нашу функцию сверху, пользуясь \hyperlink{sredneye}{теоремой о среднем}, только для простоты возьмём немного завышенный максимум: \(\left(\frac{\pi^2}4 - t^2 \right) \le 4\) при любом \(t\) из \([-\frac\pi2, \frac\pi2]\). Так как \(|\cos(t)| \le 1\), подынтегральное выражение не превосходит \(4^n\), а интеграл как интеграл константы не превосходит \(4^n \pi\). Имеем следующее: \[
		\frac{q^n}{n!} 4^n \pi \ge \frac{q^n}{n!} \int\limits_{-\frac\pi2}^{\frac\pi2}
		\left(\frac{\pi^2}4 - t^2 \right)^n \cos(t) dt \ge 1.
	\]
	Так как это справедливо для любого натурального \(n\), устремим его в бесконечность: \[
		1 \le \frac{q^n}{n!} 4^n \pi \xrightarrow[n \to \infty]{} 0.
	\]
	Получили противоречие.
\end{proof}

\subsection{Формула Тейлора с остатком в интегральной форме}

% формула Тейлора с остатком в интегральной форме
\begin{theorem}
	Пусть \(a, b \in \overline{\mathbb{R}}\), функция \(f\) дифференцируема \(n + 1\) раз на \(\langle a, b \rangle\), \(x_0 \in \langle a, b \rangle\). Тогда \[
		f(x) = \sum_{k = 0}^n \frac{f^{(k)}(x_0)}{k!} (x - x_0)^k +
		\frac{1}{n!} \int\limits_{x_0}^x (x - t)^n f^{(n + 1)}(t) dt.
	\]
\end{theorem}

\begin{proof}
	Докажем индукцией по \(n\). База ---  \(n = 0\):
	\begin{align*}
		f(x) &= f(x_0) + \int\limits_{x_0}^x f'(t) dt \\
		\int\limits_{x_0}^x f'(t) dt &= f(x) - f(x_0).
	\end{align*}
	Последнее равенство --- \hyperlink{Newton}{формула Ньютона -- Лейбница} для \(f'(x)\).
	
	Индукционный переход от \(n\) к \(n + 1\) осуществляется интегрированием по частям \(n\)-го остатка, в результате которого мы получим \((n + 1)\)-й остаток и очередное слагаемое многочлена Тейлора. Возьмём \(f^{(n + 1)}(t)\) в качестве \(f\) и \((x - t)^n\)
	в качестве \(g'\):
	\begin{gather*}
		\frac{1}{n!} \int\limits_{x_0}^x (x - t)^n f^{(n + 1)}(t) dt = \ldots \\
		\ldots = \frac{1}{n!} \left(-\frac{(x - t)^{n + 1}}{n + 1} f^{(n + 1)}(t)\right) \bigg|_{x_0}^x + \frac{1}{(n + 1)!} \int\limits_{x_0}^x (x - t)^{n + 1} f^{(n + 2)}(t) dt = \ldots \\
		\ldots = \frac{(x - x_0)^{n + 1}}{(n + 1)!} f^{(n + 1)}(x_0)
		+ \frac{1}{(n + 1)!} \int\limits_{x_0}^x (x - t)^{n + 1} f^{(n + 2)}(t) dt.
	\end{gather*}
\end{proof}

\subsection{Лемма о трёх хордах}

% лемма о трёх хордах
\begin{theorem}
	Пусть \(f \colon \langle a, b \rangle \to \mathbb{R}\). Тогда следующие утверждения эквивалентны:
	\begin{enumerate}
		\item Функция \(f\) выпукла на \(\langle a, b \rangle\),
		\item Для любых \(x_1, x_2, x_3 \in \langle a, b \rangle\), таких, что \(x_1 < x_2 < x_3\), справедливо: \[
		\frac{f(x_2) - f(x_1)}{x_2 - x_1} \le \frac{f(x_3) - f(x_1)}{x_3 - x_1} \le \frac{f(x_3) - f(x_2)}{x_2 - x_1}.
		\]
	\end{enumerate}
\end{theorem}

\begin{proof}
	Докажем равносильность сначала для левой части двойного неравенства, а потом для правой.
	Напомним определение выпуклости: \[
		f(x) \le \frac{x_3 - x}{x_3 - x_1} f(x_1) + \frac{x - x_1}{x_3 - x_1} f(x_3), \quad x \in (x_1, x_3).
	\]
	Подставим в него \(x_2\) и получим \[
		f(x_2) \le t f(x_1) + (1 - t) f(x_3),
	\]
	где \(t = \dfrac{x_3 - x_2}{x_3 - x_1}\), \(1 - t = \dfrac{x_2 - x_1}{x_3 - x_1}\).
	Преобразуем неравенство двумя способами. С одной стороны, \[
		f(x_2) \le f(x_1) + (1 - t)(f(x_3) - f(x_1)) = f(x_1) + (x_2 - x_1) \frac{f(x_3) - f(x_1)}{x_3 - x_1},
	\]
	что равносильно левой части двойного неравенства. С другой стороны, \[
		f(x_2) \le f(x_3) - t (f(x_3) - f(x_1)) = f(x_3) - (x_3 - x_2) \frac{f(x_3) - f(x_1)}{x_3 - x_1},
	\] что равносильно правой части двойного неравенства.
\end{proof}









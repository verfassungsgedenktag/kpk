\subsection{Неравенство Чебышёва}

% неравенство Чебышёва
\begin{theorem}
	Пусть \(f, g\) непрерывны на \([a, b]\) и монотонны, причём одинаково монотонны. Пусть \(I_f\) --- \hyperlink{average}{среднее значение} функции \(f\). Тогда \[
		I_f \cdot I_g \leqslant I_{fg}.
	\]
\end{theorem}

\begin{proof}
	Так как \(f, g\) монотонны одинаково, справедливо следующее: \[
		\forall x, y \in [a, b] \quad (f(x) - f(y)) \cdot (g(x) - g(y)) \geqslant 0.
	\]
	Раскроем скобки: \[
		f(x)g(x) - f(x)g(y) - f(y)g(x) + f(y)g(y) \geqslant 0.
	\]
	Проинтегрируем неравенство по \(x\) на \([a, b]\): \[
		\int_a^b fg - \int_a^b f \cdot g(y) - f(y) \cdot \int_a^b g + (b - a) \cdot f(y)g(y) \geqslant 0
	\]
	и разделим на \(b - a\): \[
		I_{fg} - I_f \cdot g(y) - f(y) \cdot I_g + f(y)g(y) \geqslant 0.
	\]
	Теперь проинтегрируем по \(y\) на том же промежутке и опять разделим на \(b - a\): \[
		I_{fg} - I_f \cdot I_g - I_f \cdot I_g + I_{fg} \geqslant 0.
	\]
	Приведём подобные и получим требуемый результат.
\end{proof}

\subsection{Иррациональность числа пи}

% иррациональность числа пи

\begin{theorem}
	Число пи иррационально.
\end{theorem}

\begin{proof}
	Рассмотрим последовательность \((H_n)\): \[
		H_n = \frac{1}{n!} \int\limits_{-\frac\pi2}^{\frac\pi2}
		\left(\frac{\pi^2}4 - t^2 \right)^n \cos(t) dt.
	\]
	Нам интересно выразить \(n\)-й член последовательности через предыдущие. Сперва проинтегрируем \(H_n\) по частям: представим подынтегральное выражение как \(fg'\),
	где \(f = \left(\frac{\pi^2}4 - t^2 \right)^n\), а \(g' = \cos(t)\). Следовательно, \(f' = -2t \cdot n \cdot \left(\frac{\pi^2}4 - t^2 \right)^{n-1}\), а \(g = \sin(t)\). Тогда \[
		H_n = \frac1{n!} \left(\frac{\pi^2}4 - t^2 \right)^n \sin(t)
		\bigg|_{-\frac\pi2}^{\frac\pi2}  - \frac2{(n - 1)!} \int_{-\frac\pi2}^{\frac\pi2}
		t \left(\frac{\pi^2}4 - t^2 \right)^{n-1} \sin(t) dt.
	\]
	Первое слагаемое зануляется, оставшееся опять проинтегрируем по частям: на этот раз
	\(f = t \left(\frac{\pi^2}4 - t^2 \right)^{n-1}\), а \(g' = \sin(t)\). Следовательно, \\
	\(f' = \left(\frac{\pi^2}4 - t^2 \right)^{n-1} - 2t^2 (n - 1) \left(\frac{\pi^2}4 - t^2 \right)^{n-2}\), а \(g = -\cos(t)\). Немного подправим \(f'\) --- во втором слагаемом вместо множителя \(t^2\)
	запишем \\
	\(\left(t^2 - \frac{\pi^2}4 \right) + \frac{\pi^2}4\), то есть \(f'\)
	примет такой вид:
	\begin{align*}
		\left(\frac{\pi^2}4 - t^2 \right)^{n - 1}
		&- 2\left(\left(t^2 - \frac{\pi^2}4 \right)
		+ \frac{\pi^2}4 \right) (n - 1) \left(\frac{\pi^2}4 - t^2 \right)^{n - 2}, \\
		\hbox{или } \left(\frac{\pi^2}4 - t^2 \right)^{n-1}
		&+ 2\left(\left(\frac{\pi^2}4 - t^2 \right)
		- \frac{\pi^2}4 \right) (n - 1) \left(\frac{\pi^2}4 - t^2 \right)^{n - 2}.
	\end{align*}
	Раскроем скобки: \[
		\left(\frac{\pi^2}4 - t^2 \right)^{n - 1}
		+ 2 (n - 1) \left(\frac{\pi^2}4 - t^2 \right)^{n - 1}
		- 2 (n - 1) \frac{\pi^2}4 \left(\frac{\pi^2}4 - t^2 \right)^{n - 2},
	\]
	и приведём подобные: \[
		(2n - 1) \left(\frac{\pi^2}4 - t^2 \right)^{n - 1}
		- (n - 1) \frac{\pi^2}2 \left(\frac{\pi^2}4 - t^2 \right)^{n - 2}.
	\]
	Запишем, наконец, результат интегрирования по частям
	\bigg(\(fg \bigg|_{-\frac\pi2}^{\frac\pi2}\) снова занулится, поэтому его писать не будем \bigg):
	\begin{align*}
		H_n &= -\frac2{(n - 1)!} \int_{-\frac\pi2}^{\frac\pi2}
		(2n - 1) \left(\frac{\pi^2}4 - t^2 \right)^{n - 1} (-\cos(t)) dt + \ldots \\
		\ldots &+ \frac2{(n - 1)!} \int_{-\frac\pi2}^{\frac\pi2}
		(n - 1) \frac{\pi^2}2 \left(\frac{\pi^2}4 - t^2 \right)^{n - 2} (-\cos(t)) dt = \ldots \\
		\ldots &= \frac{4n - 2}{(n - 1)!} \int_{-\frac\pi2}^{\frac\pi2}
		\left(\frac{\pi^2}4 - t^2 \right)^{n - 1} \cos(t) dt - \ldots \\
		\ldots &- \frac{\pi^2}{(n - 2)!} \int_{-\frac\pi2}^{\frac\pi2} \left(\frac{\pi^2}4 - t^2 \right)^{n - 2} \cos(t) dt = \ldots \\
		\ldots &= (4n - 2) H_{n - 1} + \pi^2 H_{n - 2}.
	\end{align*}
	Итак, мы нашли рекуррентную формулу для \(H_n\). Теперь мы можем выразить любой член последовательности, кроме нулевого и первого. Вычислим их непосредственно
	(при вычислении \(H_1\) придётся два раза проинтегрировать по частям):
	\begin{align*}
		H_0 &= \int_{-\frac\pi2}^{\frac\pi2} \cos(t) dt = 2, \\
		H_1 &= \int_{-\frac\pi2}^{\frac\pi2}
		\left(\frac{\pi^2}4 - t^2 \right) \cos(t) dt =
		\left(\frac{\pi^2}4 - t^2 \right) \sin(t)
		\bigg|_{-\frac\pi2}^{\frac\pi2} + \int_{-\frac\pi2}^{\frac\pi2} 2t \sin(t) dt = \ldots \\
		\ldots &= 2 \int_{-\frac\pi2}^{\frac\pi2} t \sin(t) dt = -2t \cos(t) \bigg|_{-\frac\pi2}^{\frac\pi2} + 2 \int_{-\frac\pi2}^{\frac\pi2} cos(t) dt = 4.
	\end{align*}
	Используем же, наконец, всё это, чтобы доказать, что \(\pi\) (и даже \(\pi^2\)) иррационально. Заметим, что \(H_n\) --- многочлен с целыми коэффициентами от \(\pi^2\), причём степени не больше, чем \(n\). Действительно, чтобы по полученной ранее рекурентной формуле разложить \(H_n\) до целых чисел \(H_0\) и \(H_1\), потребуется применить её \(n - 1\) раз, соответственно \(\pi^2\) может входить в получившийся в результате разложения многочлен в степени уж точно не больше, чем \(n\). Обозначим этот многочлен как \(P_n(\pi^2)\).
	
	Предположим теперь, что \(\pi^2\) --- рациональное число, то есть \hbox{\(\pi^2 = \frac{p}{q}\)}, где \(p, q \in \mathbb{N}\). Заметим, что тогда \(q^n P_n(\pi^2)\) --- целое число (так как домножением на \(q^n\) мы сократили все знаменатели). Мы также знаем, что \(P_n(\pi^2) = H_n > 0\), так как подынтегральная функция в \(H_n\) равна нулю в точках \(-\frac\pi2\) и \(\frac\pi2\) и положительна в остальных. Следовательно, \(q^n P_n(\pi^2)\) --- это как минимум единица. Запишем подробнее: \[
		q^n P_n(\pi^2) = q^n H_n = \frac{q^n}{n!} \int\limits_{-\frac\pi2}^{\frac\pi2}
		\left(\frac{\pi^2}4 - t^2 \right)^n \cos(t) dt \geqslant 1.
	\] 
	Оценим нашу функцию сверху, пользуясь \hyperlink{t7_2}{теоремой о среднем}, только для простоты возьмём немного завышенный максимум: \(\left(\frac{\pi^2}4 - t^2 \right) \leqslant 4\) при любом \(t\) из \([-\frac\pi2, \frac\pi2]\). Так как \(|\cos(t)| \leqslant 1\), подынтегральное выражение не превосходит \(4^n\), а интеграл как интеграл константы не превосходит \(4^n \pi\). Имеем следующее: \[
		\frac{q^n}{n!} 4^n \pi \geqslant \frac{q^n}{n!} \int\limits_{-\frac\pi2}^{\frac\pi2}
		\left(\frac{\pi^2}4 - t^2 \right)^n \cos(t) dt \geqslant 1.
	\]
	Так как это справедливо для любого натурального \(n\), устремим его в бесконечность: \[
		1 \leqslant \frac{q^n}{n!} 4^n \pi \xrightarrow[n \to \infty]{} 0.
	\]
	Получили противоречие.
\end{proof}

\subsection{Формула Тейлора с остатком в интегральной форме}

% формула Тейлора с остатком в интегральной форме
\begin{theorem}
	Пусть \(a, b \in \overline{\mathbb{R}}\), функция \(f\) дифференцируема \(n + 1\) раз на \(\langle a, b \rangle\), \(x_0 \in \langle a, b \rangle\). Тогда \[
		f(x) = \sum_{k = 0}^n \frac{f^{(k)}(x_0)}{k!} (x - x_0)^k +
		\frac{1}{n!} \int\limits_{x_0}^x (x - t)^n f^{(n + 1)}(t) dt.
	\]
\end{theorem}

\begin{proof}
	Докажем индукцией по \(n\). База ---  \(n = 0\):
	\begin{align*}
		f(x) &= f(x_0) + \int\limits_{x_0}^x f'(t) dt \\
		\int\limits_{x_0}^x f'(t) dt &= f(x) - f(x_0).
	\end{align*}
	Последнее равенство --- \hyperlink{t9}{формула Ньютона -- Лейбница} для \(f'(x)\).
	
	Индукционный переход от \(n\) к \(n + 1\) осуществляется интегрированием по частям \(n\)-го остатка, в результате которого мы получим \((n + 1)\)-й остаток и очередное слагаемое многочлена Тейлора. Возьмём \(f^{(n + 1)}(t)\) в качестве \(f\) и \((x - t)^n\)
	в качестве \(g'\):
	\begin{multline*}
		\frac{1}{n!} \int\limits_{x_0}^x (x - t)^n f^{(n + 1)}(t) dt = \\
		 = \frac{1}{n!} \left(-\frac{(x - t)^{n + 1}}{n + 1} f^{(n + 1)}(t)\right) \bigg|_{x_0}^x + \frac{1}{(n + 1)!} \int\limits_{x_0}^x (x - t)^{n + 1} f^{(n + 2)}(t) dt = \\
		 = \frac{(x - x_0)^{n + 1}}{(n + 1)!} f^{(n + 1)}(x_0)
		+ \frac{1}{(n + 1)!} \int\limits_{x_0}^x (x - t)^{n + 1} f^{(n + 2)}(t) dt.
	\end{multline*}
\end{proof}

\subsection{Лемма о трёх хордах}

\hypertarget{trihordy}{}
\begin{theorem}
	Пусть \(f \colon \langle a, b \rangle \to \mathbb{R}\). Тогда следующие утверждения эквивалентны:
	\begin{enumerate}
		\item Функция \(f\) выпукла на \(\langle a, b \rangle\),
		\item Для любых \(x_1, x_2, x_3 \in \langle a, b \rangle\), таких, что \(x_1 < x_2 < x_3\), справедливо: \[
		\frac{f(x_2) - f(x_1)}{x_2 - x_1} \leqslant \frac{f(x_3) - f(x_1)}{x_3 - x_1} \leqslant \frac{f(x_3) - f(x_2)}{x_3 - x_2}.
		\]
	\end{enumerate}
\end{theorem}

\begin{proof}
	Докажем равносильность сначала для левой части двойного неравенства, а потом для правой.
	Напомним определение выпуклости: \[
		f(x) \leqslant \frac{x_3 - x}{x_3 - x_1} f(x_1) + \frac{x - x_1}{x_3 - x_1} f(x_3), \quad x \in (x_1, x_3).
	\]
	Подставим в него \(x_2\) и получим \[
		f(x_2) \leqslant t f(x_1) + (1 - t) f(x_3),
	\]
	где \(t = \dfrac{x_3 - x_2}{x_3 - x_1}\), \(1 - t = \dfrac{x_2 - x_1}{x_3 - x_1}\).
	Преобразуем неравенство двумя способами. С одной стороны, \[
		f(x_2) \leqslant f(x_1) + (1 - t)(f(x_3) - f(x_1)) = f(x_1) + (x_2 - x_1) \frac{f(x_3) - f(x_1)}{x_3 - x_1},
	\]
	что равносильно левой части двойного неравенства. С другой стороны, \[
		f(x_2) \leqslant f(x_3) - t (f(x_3) - f(x_1)) = f(x_3) - (x_3 - x_2) \frac{f(x_3) - f(x_1)}{x_3 - x_1},
	\] что равносильно правой части двойного неравенства.
\end{proof}

\subsection{Теорема об односторонней дифференцируемости выпуклой функции}

\begin{theorem}[версия Виноградова]
	Пусть функция \(f\) выпукла вниз на \(\langle a, b \rangle\). Тогда для любой точки \(x \in (a, b)\) существуют конечные \(f'_-(x)\) и \(f'_+(x)\), причём \(f'_-(x) \leqslant f'_+(x)\).
\end{theorem}

\begin{proof}
	Возьмём \(x \in (a, b)\) и положим \[
		g(\xi) = \frac{f(\xi) - f(x)}{\xi - x}, \qquad \xi \in \langle a, b \rangle \setminus \{x\}.
	\]
	
	По \hyperlink{trihordy}{лемме о трёх хордах} \(g\) возрастает на \(\langle a, b \rangle \setminus \{x\}\). Поэтому если \hbox{\(a < \xi < x < \eta < b\)}, то \[
		\frac{f(\xi) - f(x)}{\xi - x} \leqslant \frac{f(\eta) - f(x)}{\eta - x}.
	\]
	Следовательно, \(g\) ограничена на \(\langle a, x)\) сверху, а на \((x, b \rangle\) --- снизу. По теореме о пределе монотонной функции существуют конечные пределы \(g(x-)\) и \(g(x+)\), которые по определению являются односторонними производными \(f'_-(x)\) и \(f'_+(x)\). Устремляя \(\xi\) к \(x\) слева, а \(\eta\) к \(x\) справа, получаем, что \(f'_-(x) \leqslant f'_+(x)\). 
\end{proof}

\begin{theorem}[версия Кохася]
	Пусть функция \(f\) выпукла вниз на \(\langle a, b \rangle\). Тогда для любой точки \(x \in (a, b)\) существуют конечные \(f'_-(x)\) и \(f'_+(x)\), причём \(\forall x_1, x_2 \in (a, b)\), где \(x_1 < x_2\), справедливо \[
		f'_- (x_1) \leqslant f'_+ (x_1) \leqslant \frac{f(x_2) - f(x_1)}{x_2 - x_1} \leqslant f'_- (x_2) \leqslant f'_+ (x_2).
	\]
\end{theorem}
\begin{proof}
	Проведём те же самые рассуждения, а потом найдем левую и правую производные в точках \(x_1, x_2\). Получим \[
		f'_- (x_1) \leqslant f'_+ (x_1) \leqslant f'_- (x_2) \leqslant f'_+ (x_2).
	\]
	Так как \(f'_- (x_2)\) --- это предел \(g(x_2-)\), а \(f'_+ (x_1)\) --- предел \(g(x_1+)\), а также из того, что \(x_1 < x_2\), очевидно, что \[
	f'_- (x_1) \leqslant f'_+ (x_1) \leqslant \frac{f(x_2) - f(x_1)}{x_2 - x_1} \leqslant f'_- (x_2) \leqslant f'_+ (x_2).
	\]
\end{proof}

\subsection{Описание выпуклости с помощью касательных}

\hypertarget{vypkas}{}
\begin{theorem}
	Пусть функция \(f\) дифференцируема на \(\langle a, b \rangle\). Тогда \(f\) выпукла вниз на \(\langle a, b \rangle\) в том и только том случае, когда график \(f\) лежит не ниже любой своей касательной, то есть для любых \(x, x_0 \in \langle a, b \rangle\): \[
		f(x) \geqslant f(x_0) + f'(x_0) (x - x_0). \eqno{(*)}
	\]
\end{theorem}

\begin{proof}
	Докажем необходимость и достаточность отдельно.
	\begin{enumerate}
		\item[\(\Rightarrow\)] Пусть \(f\) выпукла вниз, \(x, x_0 \in \langle a, b \rangle\). Если \(x > x_0\), то по \hyperlink{trihordy}{лемме о трёх хордах} для любого \(\eta \in (x_0, x)\) \[
			\frac{f(\eta) - f(x_0)}{\eta - x_0} \leqslant \frac{f(x) - f(x_0)}{x - x_0}.
		\]
		Устремляя \(\eta\) к \(x_0\), получаем неравенство \[
			f'(x_0) \leqslant \frac{f(x) - f(x_0)}{x - x_0},
		\]
		равносильное~(\textasteriskcentered).
		
		Если \(x < x_0\), то по \hyperlink{trihordy}{лемме о трёх хордах} для любого \(\xi \in (x, x_0)\) \[
			\frac{f(\xi) - f(x_0)}{\xi - x_0} \geqslant \frac{f(x) - f(x_0)}{x - x_0}.
		\]
		Устремляя \(\xi\) к \(x_0\), получаем неравенство \[
			f'(x_0) \geqslant \frac{f(x) - f(x_0)}{x - x_0},
		\]
		равносильное~(\textasteriskcentered).
		\item[\(\Leftarrow\)] Пусть для любых \(x, x_0 \in \langle a, b \rangle\) верно неравенство~(\textasteriskcentered). Возьмём \(x_1, x_2 \in \langle a, b \rangle \mid x_1 < x_2\), и \(x \in (x_1, x_2)\). Применяя неравенство~(\textasteriskcentered) дважды: сначала к точкам \(x_1\) и \(x\), а затем --- к \(x_2\) и \(x\), получаем \[
			f(x_1) \geqslant f(x) + f'(x) (x_1 - x), \qquad f(x_2) \geqslant f(x) + f'(x) (x_2 - x),
		\]
		что равносильно \[
			\frac{f(x_1) - f(x)}{x_1 - x} \leqslant f'(x) \leqslant \frac{f(x_2) - f(x)}{x_2 - x}.
		\]
		Крайние части составляют неравенство, равносильное неравенству из определения выпуклости. Действительно, домножим обе части на знаменатели, выразим \(f(x)\) и получим требуемое неравенство.
	\end{enumerate}
\end{proof}

\subsection{Дифференциальные критерии выпуклости}

\begin{theorem}
	Пусть функция \(f\) непрерывна на \(\langle a, b \rangle\). Тогда:
	\begin{enumerate}
		\item Если \(f\) дифференцируема на \((a, b)\), то \(f\) (строго) выпукла вниз на \(\langle a, b \rangle\) в том и только том случае, когда \(f'\) (строго) возрастает на \(\langle a, b \rangle\).
		\item Если \(f\) дважды дифференцируема на \((a, b)\), то \(f\) выпукла вниз в том и только том случае, когда \(f''(x) \geqslant 0\) для всех \(x \in (a, b)\).
	\end{enumerate}
\end{theorem}

\begin{proof}
	Докажем критерии по отдельности.
	\begin{enumerate}
		\item Докажем сперва необходимость, а потом достаточность.
		\begin{enumerate}
			\item[\(\Rightarrow\)] \label{dif}Возьмём \(x_1, x_2 \in (a, b) \mid x_1 < x_2\). Согласно \hyperlink{vypkas}{описанию выпуклости с помощью касательных}, имеем
			\begin{equation}
				\label{dif1}
				f'(x_1) \leqslant \frac{f(x_2) - f(x_1)}{x_2 - x_1} \leqslant f'(x_2).
			\end{equation}
			Действительно, сначала запишем неравенство для касательной в точке \(x_1\) и подставим в него \(x_2\), получив левую часть, а потом запишем неравенство для касательной в точке \(x_2\) и подставим в него \(x_1\), получив правую часть. Но ведь получившееся неравенство и означает возрастание \(f'\).
			\item[\(\Leftarrow\)] Возьмём \(x_1, x_2 \in (a, b) \mid x_1 < x_2\) и \(x \in (x_1, x_2)\). По теореме Лагранжа существуют такие \(c_1 \in (x_1, x)\) и \(c_2 \in (x, x_2)\), что \[
				\frac{f(x) - f(x_1)}{x - x_1} = f'(c_1), \qquad \frac{f(x_2) - f(x)}{x_2 - x} = f'(c_2).
			\]
			Тогда \(x_1 < c_1 < x < c_2 < x_2\), а \(f'\) по условию возрастает, поэтому \(f'(c_1) \leqslant f'(c_2)\), то есть
			\begin{equation}
				\label{dif2}
				\frac{f(x) - f(x_1)}{x - x_1} \leqslant \frac{f(x_2) - f(x)}{x_2 - x},	
			\end{equation}
			что равносильно неравенству из определения выпуклости (аналогично ситуации в доказательстве \hyperlink{vypkas}{описания выпуклости с помощью касательных}).
		\end{enumerate}
		
		Если \(f\) строго выпукла вниз, то оба неравенства в \eqref{dif1} строгие. Обратно, если \(f'\) строго возрастает, то неравенство \eqref{dif2} строгое, что влечёт строгую выпуклость \(f\).
		
		\item По пункту 1 выпуклость \(f\) равносильна возрастанию \(f'\), которое по критерию монотонности равносильно неотрицательности \(f''\).
	\end{enumerate}
\end{proof}

\subsection{\itshape Теорема о вычислении аддитивной функции промежутка по плотности}

\hypertarget{afp}{}
\begin{theorem}
	Пусть \(\Phi \colon Segm \langle a, b \rangle \to \mathbb{R}\) --- аддитивная функция промежутка, \(f \colon \langle a, b \rangle \to \mathbb{R}\) --- плотность \(\Phi\). Тогда \[
		\forall [p, q] \in Segm \langle a, b \rangle \quad \Phi([p, q]) = \int_p^q f.
	\]
\end{theorem}

\begin{proof}
	Зафиксируем \([p, q] \in Segm \langle a, b \rangle\). Рассмотрим функцию \(F\): \[
		F(x) =
		\begin{cases}
			\Phi([p, x]), & p < x \leqslant q \\
			0,			  & x = p.
		\end{cases}
	\]
	
	Теперь проверим, что \(F\) --- первообразная \(f\) на \([p, q]\). Имеем: \[
		F'_+(x) = \lim_{h \to +0} \frac{\Phi([p, x + h]) - \Phi([p, x])}{h} = \lim_{h \to +0} \frac{\Phi([x, x + h])}{h}.
	\]
	По определению плотности
	\begin{gather*}
		h \cdot \inf_{[x, x + h]}(f) \leqslant \Phi([x, x + h]) \leqslant h \cdot \sup_{[x, x + h]}(f) \\
		\inf_{[x, x + h]}(f) \leqslant \frac{\Phi([x, x + h])}{h} \leqslant \sup_{[x, x + h]}(f).
	\end{gather*}
	По теореме Вейерштрасса инфимум и супремум функции на отрезке достигаются, то есть неравенство равносильно \[
		\min_{[x, x + h]}(f) \leqslant \frac{\Phi([x, x + h])}{h} \leqslant \max_{[x, x + h]}(f).
	\]
	Перейдём к пределу при \(h \to +0\): \[
		f(x) \leqslant \frac{\Phi([x, x + h])}{h} \leqslant f(x).
	\]
	
	Аналогично \(F'_-(x) = f(x)\). Применим \hyperlink{t9}{формулу Ньютона -- Лейбница} и получим требуемый результат: \[
		\Phi([p, q]) = F(q) - F(p) = \int_p^q f.
	\]
	ДОДЕЛАТЬ ПРО ТЕОРЕМУ О ПРОМЕЖУТОЧНОМ ЗНАЧЕНИИ И КОНЦЫ ПРОМЕЖУТКА
\end{proof}

\subsection{Свойства верхнего и нижнего пределов}

\begin{theorem}
	Пусть \((x_n)\) --- вещественная последовательность. Тогда\footnote{Все пределы обсуждаются при \(n \to \infty\).}:
	\begin{enumerate}
		\item \(\varliminf\limits_{n \to \infty} x_n \leqslant \varlimsup\limits_{n \to \infty} x_n\),
		\item Если \((\widetilde{x}_n)\) --- вещественная последовательность, причём для всех \(n\) \(x_n \leqslant \widetilde{x}_n\), то \(\varlimsup\limits_{n \to \infty} x_n \leqslant \varlimsup\limits_{n \to \infty} \widetilde{x}_n\) и \(\varliminf\limits_{n \to \infty} x_n \leqslant \varliminf\limits_{n \to \infty} \widetilde{x}_n\),
		\item Возьмём \(\lambda \geqslant 0\), тогда \(\varlimsup\limits_{n \to \infty} \lambda x_n = \lambda \varlimsup\limits_{n \to \infty} x_n\) и \(\varliminf\limits_{n \to \infty} \lambda x_n = \lambda \varliminf\limits_{n \to \infty} x_n\),\footnote{Заметим, что при этом выходит, что \(0 \cdot (\pm \infty) = 0\).}
		\item \(\varlimsup\limits_{n \to \infty} -x_n = -\varliminf\limits_{n \to \infty} x_n\), \(\varliminf\limits_{n \to \infty} -x_n = -\varlimsup\limits_{n \to \infty} x_n\),
		\item \(\varlimsup\limits_{n \to \infty} x_n + \widetilde{x}_n \leqslant \varlimsup\limits_{n \to \infty} x_n + \varlimsup\limits_{n \to \infty} \widetilde{x}_n\) и \(\varliminf\limits_{n \to \infty} x_n + \widetilde{x}_n \geqslant \varliminf\limits_{n \to \infty} x_n + \varliminf\limits_{n \to \infty} \widetilde{x}_n\), где \((\widetilde{x}_n)\) --- вещественная последовательность,
		\item Если \((t_n)\) --- вещественная последовательность, стремящаяся к \(l \in \mathbb{R}\), то \(\varlimsup\limits_{n \to \infty} x_n + t_n = \varlimsup\limits_{n \to \infty} x_n + l\),
		\item Если \((t_n)\) --- вещественная последовательность, стремящаяся к \(l \in (0, +\infty)\), то \(\varlimsup\limits_{n \to \infty} t_n \cdot x_n = l \cdot \varlimsup\limits_{n \to \infty} x_n\).
	\end{enumerate}
\end{theorem}

\begin{proof}
	Докажем эти свойства:
	\begin{enumerate}
		\item Следует из того, что для всех \(n\) \(z_n \leqslant y_n\),
		\item В силу того, что для всех \(n\) \(y_n \leqslant \widetilde{y}_n\) и \(z_n \leqslant \widetilde{z}_n\),
		\item Так как для \((\lambda x_n)\) верхняя огибающая имеет вид \((\lambda y_n)\), а нижняя~--- \((\lambda z_n)\),
		\item Обозначим как \((y_n^-)\) верхнюю огибающую \((-x_n)\) и как \((z_n^-)\) --- нижнюю. Тогда \[
			y_n^- = \sup(-x_n, -x_{n + 1}, -x_{n + 2}, \ldots).
		\]
		По определению супремума \[
			\forall k \geqslant n \quad y_n^- \geqslant -x_k, \ \text{то есть} \ -y_n^- \leqslant x_k.
		\]
		Получается, что \(-y_n^- = \inf(x_n, x_{n + 1}, x_{n + 2}, \ldots)\), что равносильно \linebreak \(y_n^- = -\inf(x_n, x_{n + 1}, x_{n + 2}, \ldots) = -z_n\). Аналогично \(z_n^- = -y_n\).
		
		Перейдём к пределу и получим требуемый результат.
		\item ДОДЕЛАТЬ
		\item По условию \(t_n \to l \in \mathbb{R}\), то есть \[
			\forall \varepsilon > 0 \quad \exists N_0 \quad \forall n > N_0 \quad l - \varepsilon < t_n < l + \varepsilon.
		\]
		Прибавим к неравенству \(x_n\): \[
			\forall \varepsilon > 0 \quad \exists N_0 \quad \forall n > N_0 \quad x_n + l - \varepsilon < x_n + t_n < x_n + l + \varepsilon.
		\]
		Перейдём к супремуму в неравенстве (супремумом \(x_n\) является \(y_{N}\), где \(N_0 < N < n\)): \[
			y_N + l - \varepsilon < \sup(x_n + t_n, x_{n + 1} + t_{n + 1}, x_{n + 2} + t_{n + 2}, \ldots) < y_N + l + \varepsilon.
		\]
		Устремим \(N\), а значит, и \(n\) тоже в бесконечность, а \(\varepsilon\) --- к нулю, и получим нужный результат: \[
			\varlimsup\limits_{n \to \infty} x_n + l \leqslant \varlimsup\limits_{n \to \infty} x_n + t_n \leqslant \varlimsup\limits_{n \to \infty} x_n + l.
		\]
		\item Без доказательства.
	\end{enumerate}
\end{proof}

\subsection{Техническое описание верхнего предела}

\hypertarget{техническое описание верхнего предела}{}
\begin{theorem}
	Пусть \((x_n)\) --- вещественная последовательность. Тогда:
	\begin{enumerate}
		\item \(\varlimsup\limits_{n \to \infty} x_n = +\infty\) \(\Leftrightarrow\) \((x_n)\) не ограничена сверху,
		\item \(\varlimsup\limits_{n \to \infty} x_n = -\infty\) \(\Leftrightarrow\) \(x_n \to -\infty\),
		\item \(\varlimsup\limits_{n \to \infty} x_n = l\) тогда и только тогда, когда выполняются условия:
		\begin{enumerate}
			\item \[
				\forall \varepsilon > 0 \quad \exists N_0 \quad \forall n > N_0 \quad x_n < l + \varepsilon,
			\]
			\item \[
				\forall \varepsilon > 0 \quad \exists \textit{бесконечно много} \ n \quad l - \varepsilon < x_n.
			\]
		\end{enumerate}
	\end{enumerate}
\end{theorem}

\begin{proof}
	Докажем пункты по отдельности:
	\begin{enumerate}
		\item Очевидно: \[
			\varlimsup_{n \to \infty} x_n = +\infty \Leftrightarrow \lim_{n \to \infty} y_n = +\infty \Leftrightarrow \forall E > 0 \enskip \exists n \enskip E < y_n.
		\]
		А так как \(y_n = \sup(x_n, x_{n + 1}, x_{n + 2}, \ldots)\), выражение выше равносильно тому, что \((x_n)\) не ограничена сверху.
		\item 
		\begin{enumerate}
			\item[\(\Rightarrow\)] Тоже очевидно: \[
			\varlimsup\limits_{n \to \infty} x_n = -\infty \Leftrightarrow \lim_{n \to \infty} y_n = -\infty.
			\]
			В силу того, что для всех \(n\) \(x_n \leqslant y_n\), \(x_n \to -\infty\),
			\item[\(\Leftarrow\)] При \(n \to \infty\) справедливо: \[
				x_n \to -\infty \Rightarrow \sup(x_n, x_{n + 1}, x_{n + 2}, \ldots) \to -\infty \Leftrightarrow y_n \to -\infty,
			\]
			что и означает, что \(\varlimsup\limits_{n \to \infty} x_n\).
		\end{enumerate}
		\item
		\begin{enumerate}
			\item[\(\Rightarrow\)] Имеем, что \(y_n \to l\) при \(n \to \infty\), то есть \[
				\forall \varepsilon > 0 \quad  \exists N_0 \quad \forall n > N_0 \quad l - \varepsilon < y_n < l + \varepsilon.
			\]
			Так как для всех \(n\) \(x_n \leqslant y_n\), получаем, что \[
				\forall \varepsilon > 0 \quad  \exists N_0 \quad \forall n > N_0 \quad x_n < l + \varepsilon,
			\]
			о чём и говорится в пункте (a). Пункт (b) следует из того, что \(y_n = \sup(x_n, x_{n + 1}, x_{n + 2}, \ldots)\) стремится к \(l\).
			\item[\(\Leftarrow\)] Передём к супремуму в имеющихся неравенствах (напомним, что супремумом \(x_n\) при \(n \to \infty\) является \(y_N\), где \(N_0 < N < n\)). Получим, что \(l - \varepsilon < y_N < \ + \varepsilon\) для любого \(\varepsilon > 0\) (нижняя граница верна, потому что \(y_N\) --- супремум обсуждаемого в условии бесконечного множества \(x_n\)). Устремим \(N\) в бесконечность и получим требуемый результат.
		\end{enumerate}
	\end{enumerate}
\end{proof}